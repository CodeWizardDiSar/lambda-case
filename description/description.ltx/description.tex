\documentclass{article}

\usepackage[utf8]{inputenc}
\usepackage{amsmath}
\usepackage{fancyvrb}
\usepackage{graphicx}
\usepackage{syntax}
\usepackage{array}
\usepackage{multirow}
\usepackage{mathtools}
\usepackage{fancyvrb}
\usepackage[left=2cm, right=2cm, top=2cm]{geometry}

\usepackage{biblatex}
\addbibresource{parsec.bib}

\usepackage{hyperref}
\hypersetup{
  colorlinks=true, linkcolor=blue, filecolor=magenta, urlcolor=cyan
}

\date{}
\author{ Dimitris Saridakis }

\def\imgs{../Images}
\def\H{Haskell}
\def\r{$\Longleftrightarrow$\ }
\def\lra{$\Longrightarrow$\ }
\def\pend{\mbox{}\\\\}
\def\prarr{$\xRightarrow{\text{preprocessing}}$\ }
%\newcommand\_[_]{_}
\newcommand{\mytilde}{$\sim$}

\renewcommand{\arraystretch}{1.5}

\begin{document}

\title{
\textbf{Lambda Cases (lcases)}
}
\maketitle

\tableofcontents

\newpage
\section{Introduction}

\H\ is a delightful language. Yet, it doesn't seem to have its rightful place
in terms of popularity in industry. Why is it so?  Is it inherently hard to
learn and therefore only the brave dare to use it, or could it be that the
syntax is perplexing to the amateur eye? It is my belief that with some syntax
changes that give a greater familiarity to the new user, there would be no
language more compelling than (the new) \H. In an attempt to achieve that
familiarity, I present some (hopefully useful) new syntax, of which some is
closer to the imperative/OOP style (to attract more already experienced
programmers from these languages), some is closer to mathematics (in which most
programmers should be experienced) and some is closer to natural language (in
which we are all already experienced).

\section{Language Description: General}

\subsection{Program Structure}

An lcases program consists of a set of definitions, type nicknames and
theorems.  Definitions are split into value definitions, type definitions and
type proposition definitions. Theorems are proven type propositions. Functions
as well as "Environment Actions" (see section \ref{subsec:envacts}) are also
considered values. The definition of the "main" value determines the program's
behaviour.

\paragraph{Program example: Euclidean Algorithm}
\begin{verbatim}
gcd_of(_)and(_): Int^2 => Int
  = (x, cases)
    0 => x
    y => gcd_of(y)and((x)mod(y))

read_two_ints: (Int^2)WithIO
  = print_line("Please give me 2 ints");
    get_line ;> split(_)to_words o> cases
      [x, y] => (from_string(x), from_string(y)) -> (_)with_io
      ... => throw_err("You didn't give me 2 ints")

tuple_type NumsAndGcd
value (x, y, gcd):Int^3

nag(_)to_message: NumsAndGcd => String
  = nag => "The GCD of " + nag.x + " and " + nag.y + " is " + nag.gcd

main: IO
  = read_two_ints ;> (i1, i2) =>
    (i1, i2, gcd_of(i1)and(i2)) -> nag(_)to_message -> print_line(_)
\end{verbatim}

\paragraph{Program grammar}
\begin{grammar}
<program> ::=
<nl>* <program-part> ( <nl> <nl> <program-part> )* <nl>*

<program-part> ::=
<value-def> | <grouped-value-defs> | <type-def> | <t-nickname> |
<type-prop-def> | <type-theo>

<nl> :: ( `\ ' | `\\t' )* `\\n'
\end{grammar}

\newpage
\subsection{Keywords}

The lcases keywords are the following:
\begin{verbatim}
cases all where tuple_type value or_type values
type_proposition needed equivalent type_theorem proof
\end{verbatim}
Each keyword's functionality is described in the respective section shown in
the table below:

\begin{center}
\begin{tabular}{ |c|c| }
\hline
Keyword & Section
\\
\hline
\hline
\verb|cases| & \ref{subsubsec:casessyntax} "cases" Function Expressions
\\
\hline
\verb|all where| & \ref{subsec:valdefswhere}
Value Definitions and "where" Expressions
\\
\hline
\verb|tuple_type value or_type values type_nickname| &
\ref{subsec:types} Types
\\
\hline
\verb|type_proposition needed equivalent type_theorem proof| &
{\ref{subsec:typelogic} Type Logic}
\\
\hline
\end{tabular}
\end{center}
The "\verb|cases|" and "\verb|where|" keywords are also
reserved words. Therefore, even though they can be generated by the
"identifiers" grammar, they cannot be used as identifiers (see "Literals and
Identifiers" section \ref{subsubsec:litsandidents}).

\newpage
\section{Language Description: Values}

\subsection{Basic Expressions}

\subsubsection{Literals and Identifiers}
\label{subsubsec:litsandidents}

\paragraph{Literals}
\begin{itemize}

\item \textit{Examples}
\begin{verbatim}
1  2  17  42  -100
1.62  2.72  3.14  -1234.567
'a'  'b'  'c'  'x'  'y'  'z'  '.'  ','  '\n'
"Hello World!"  "What's up, doc?"  "Alrighty then!"
\end{verbatim}

\item \textit{Description}

There are literals for the four basic types: \texttt{Int, Real, Char, String}.

\item \textit{Grammar}
\begin{grammar}
<literal> ::= <int-lit> | <real-lit> | <char-lit> | <string-lit>
\end{grammar}

\end{itemize}


\paragraph{Identifiers}
\begin{itemize}

\item \textit{Examples}
\begin{verbatim}
x y z
a1 a2 a3
(_)mod(_)
apply(_)to_all_in(_)
\end{verbatim}

\item \textit{Description}

An identifier is the name of a value or a parameter. It is used in the
definition of a value and in expressions that use that value, or in the
parameters of a function and in the body of that function.
\\\\
An identifier starts with a lower case letter, which can followed by lower case
letters or underscores and/or ended with a digit. It is also possible to have
underscores in parenthesis before, after or in the middle of an identifier (see
"Parenthesis Function Application" section \ref{subsubsec:parenfuncapp} for why
this can be useful).
\\\\
A simple identifier is an identifier without any underscores in parenthesis.
It used in expressions where underscores in parenthesis don't make sense (e.g.
"Prefix and Postfix Functions" \ref{subsubsec:prefixpostfix}).

\item \textit{Grammar}
\begin{grammar}
<identifier> ::=
[ <unders-in-paren> ] <id-start> <id-cont>* [ [0-9] ] [ <unders-in-paren> ]

<simple-id> ::= <id-start> [ [0-9] ]

<id-start> ::= [a-z] [a-z_]*

<id-cont> ::= <unders-in-paren> [a-z_]+

<unders-in-paren> ::= `(_' ( <comma> `_' )* `)'

<comma> ::= `,' [ `\ ' ]
\end{grammar}
Even though the "\verb|cases|" and "\verb|where|" keywords can be generated
by these grammar rules, they cannot be used as identifiers.

\end{itemize}

\newpage
\subsubsection{Parenthesis, Tuples and Lists}

\paragraph{Parenthesis}

\begin{itemize}

\item \textit{Examples}
\begin{verbatim}
(1 + 2)
(((1 + 2) * 3)^4)
(n => 3*n + 1)
(get_line ;> line => print("Line is: " + line))
\end{verbatim}

\item \textit{Description}

An expression is put in parenthesis to prioritize it or isolate it in a bigger
(operator) expression. The expressions inside parenthesis are operator
or function expressions.
\\\\
Parenethesis expressions cannot extend over multiple lines. For expressions
that extend of over multiple lines new values must be defined.

\item \textit{Grammar}
\begin{grammar}
<paren-expr> ::= `(' [ `\ ' ] <line-op-expr> | <line-func-expr> [ `\ ' ] `)'
\end{grammar}

\end{itemize}

\paragraph{Tuples}

\begin{itemize}

\item \textit{Examples}
\begin{verbatim}
(1, "What's up, doc?")
(2, "Alrighty then!", 3.14)
(x, y, z, w)
(1, my_function, (x, y, z) => sqrt(x^2 + y^2 + z^2))
\end{verbatim}

\item \textit{Description}

Tuples are used to group many values (of possibly different types) into one.
The type of a tuple can be either the product of the types of the fields or a
defined \verb|tuple_type| which is equivalent to the afformentioned product
type (see "Tuple Types" in section \ref{subsubsec:tdefs} for details). For
example, the type of the second tuple above could be:
\begin{verbatim}
Int x String x Real
\end{verbatim}
or:
\begin{verbatim}
MyType
\end{verbatim}
assuming "\verb|MyType|" has been defined in a similar way to the following:
\begin{verbatim}
tuple_type MyType
value
  (my_int, my_string, my_real) : Int x String x Real
\end{verbatim}

\newpage
\item \textit{Big Tuples}
\\\\
\textbf{Example}
\begin{verbatim}
my_big_tuple
  : String x Int x Real x String x String x (String x Real x Real)
  = ( "Hey, I'm the first field and I'm also a relatively big string."
    , 42, 3.14, "Hey, I'm the first small string", "Hey, I'm the second small string"
    , ("Hey, I'm a string inside the nested tuple", 2.72, 1.62)
    )
\end{verbatim}

\textbf{Description}

It is possible to stretch a (big) tuple expression over multiple lines (only)
in a seperate value definition (see "Value Definitions" section
\ref{subsubsec:valdefs}).  In that case:
\begin{itemize}
\item
The character '\verb|(|' is after the "\verb|=| " part of the value definition
and the first field must be in the same line.

\item
The tuple can split in a new line only at a '\verb|,|' character. Every such
line must be indented so that the '\verb|,|' is in same column where the
'\verb|(|' character was in the first line.

\item
The tuple must be ended by a line that only contains the '\verb|)|' character
and is also indented so that the '\verb|)|' is in same column where the
'\verb|(|' character was in the first line.

\item
The precise indentation rules are described in the section
"Indentation System" \ref{subsubsec:indsys}.
\end{itemize}

\item \textit{Tuples with empty fields}
\\\\
\textbf{Example}
\begin{verbatim}
(42, _)
(_, 3.14, _)
(_, _, "Hello from 3rd field")
\end{verbatim}

\textbf{Description}\\\\
It is possible to leave some fields empty in a tuple by having an underscore in
their position. This creates a function that expects the empty fields
and returns the whole tuple. This is best demonstated by the types of the
examples above:
\begin{verbatim}
(42, _) : T1 => Int x T1
(_, 3.14, _) : T1 x T2 => T1 x Real x T2
(_, _, "Hello from 3rd field") : T1 x T2 => T1 x T2 x String
\end{verbatim}
An example in a bigger expression is the following:
\begin{verbatim}
questions : ListOf(String)s
  = ["the Ultimate Question of Life", "the Universe", "Everything"]

answers_to : ListOf(String)s
  = apply("The answer to " + _)to_all_in(questions)

>> apply((42, _))to_all_in(answers_to)
  : ListOf(Int x String)s
  ==> [ (42, "The answer to the Ultimate Question of Life")
      , (42, "The answer to the Universe")
      , (42, "The answer to Everything")
      ]
\end{verbatim}

\item \textit{Grammar}
\begin{grammar}
<tuple> ::=
`(' [ `\ ' ] <line-expr-or-under> <comma> <line-expr-or-unders> [ `\ ' ] `)'

<line-expr-or-unders> ::=
<line-expr-or-under> ( <comma> <line-expr-or-under> )*

<line-expr-or-under> ::= <line-expr> | `_'

<line-expr> ::= <basic-or-app-expr> | <line-op-expr> | <line-func-expr>

<basic-or-app-expr> ::= <basic-expr> | <pre-func-app> | <post-func-app>

<basic-expr> ::=
<literal> | <paren-func-app-or-id> | <special-id> | <tuple> | <list>
\\

<big-tuple> ::= ""\\
`(' [ `\ ' ] <line-expr-or-under> [ <nl> <indent> ]
<comma> <line-expr-or-unders> \\
( <nl> <indent> <comma> <line-expr-or-unders> )*
<nl> <indent> `)'
\end{grammar}

\end{itemize}

\paragraph{Lists}

\begin{itemize}

\item \textit{Examples}
\begin{verbatim}
[1, 2, 17, 42, -100]
[1.62, 2.72, 3.14, -1234.567]
["Hello World!", "What's up, doc?", "Alrighty then!"]
[x => x + 1, x => x + 2, x => x + 3]
[x, y, z, w]
\end{verbatim}

\item \textit{Description}

Lists are used to group many values of the same type into one.  The type of the
list is \verb|ListOf(T1)s| where \verb|T1| is the type of every value inside.
Therefore, the types of the first four examples are:
\begin{verbatim}
ListOf(Int)s
ListOf(Real)s
ListOf(String)s
(@A)And(Int)Add_To(@B) --> ListOf(@A => @B)s
\end{verbatim}
And the last list is only legal if x, y, z and w all have the same type.
Assuming they do and it's the type T, the type of the list is:
\begin{verbatim}
ListOf(T)s
\end{verbatim}

\newpage
\item \textit{Big Lists}
  \\\\
  \textbf{Example}

  \begin{verbatim}
my_big_list: ListOf(Int => IO)s
  = [ x => print("I'm the first function and x + 1 is: " + (x + 1))
    , x => print("I'm the second function and x + 2 is: " + (x + 2))
    , x => print("I'm the third function and x + 3 is: " + (x + 3))
    ]
  \end{verbatim}

  \textbf{Description}

  It is possible to stretch a (big) list expression over multiple lines (only)
  in a seperate value definition (see "Value Definitions" section
  \ref{subsubsec:valdefs}).  In that case:
  \begin{itemize}
  \item
  The character '\verb|[|' is after the "= " part of the value definition
  and the first element must be in the same line.

  \item
  The list can split in a new line only at a '\verb|,|' character. Every such
  line must be indented so that the '\verb|,|' is in same column where the
  '\verb|[|' character was in the first line.

  \item
  The list must be ended by a line that only contains the '\verb|]|' character
  and is also indented so that the '\verb|]|' is in same column where the
  '\verb|[|' character was in the first line.

  \item
  The precise indentation rules are described in the section
  "Indentation System" \ref{subsubsec:indsys}.
  \end{itemize}

\item \textit{Grammar}
\begin{grammar}
<list> ::= `[' [ `\ ' ] [ <line-expr-or-unders> ] [ `\ ' ] `]'

<big-list> ::= ""\\
`[' [ `\ ' ] <line-expr-or-unders>
( <nl> <indent> <comma> <line-expr-or-unders> )* <nl> <indent> `]'
\end{grammar}

\end{itemize}

\newpage
\subsubsection{Parenthesis Function Application}
\label{subsubsec:parenfuncapp}

\begin{itemize}
\item \textit{Examples}

\begin{verbatim}
f(x)
f(x, y, z)
(x)to_string
apply(f)to_all_in(l)
\end{verbatim}

\item \textit{Description}

Function application in lcases can be done in many different ways. In this
section, we discuss the ways function application can be done with parenthesis.
\\\\
In the first two examples, we have the usual mathematical function application
which is also used in most programming languages and should be familiar to the
reader, i.e. function application is done with the arguments of the function in
parenthesis seperated by commas and \textbf{appended} to the function
identifier.
\\\\
This idea can be extended by allowing the arguments to be \textbf{prepended} or
to be \textbf{inside} to the function identifier (examples 3 and 4). This is
only valid if the function has been \textbf{defined with these parentheses in
the identifier}. For example, below is the definition of
"\verb|apply(_)to_all_in(_)|":

\begin{verbatim}
apply(_)to_all_in(_) : (T1 => T2) x ListOf(T1)s => ListOf(T2)s
  = (f, cases)
    empty_l => empty_l
    non_empty_l:l => non_empty_l:(f <- l.head, apply(f)to_all_in(l.tail))
\end{verbatim}
The actual definition doesn't matter at this point, what matters is that the
identifier is "\verb|apply(_)to_all_in(_)|" with the parentheses
\textbf{included}.  This is very useful for defining functions where the
argument in the middle or before makes the function application look and sound
more like natural language.
\\\\
It is possible to have many parentheses in a single function application (last
example). The arguments are always inserted to the function from \textbf{left
to right}.  Therefore, when multiple parentheses are present the arguments of
the leftmost parentheses are inserted first then the next ones to the right and
so on.

\newpage
\item \textit{Empty arguments in Parenthesis Function Application}
\\\\
It is possible to give a function only some of the arguments by putting an
underscore to all the missing arguments. The resulting expression is a function
that expects the missing arguments to return the final result. Let's see
this in action:
\begin{verbatim}
f : Char x Int x Real => String
c, i, r : Char, Int, Real

f(c, i, r) : String

f(_, i, r) : Char => String
f(c, _, r) : Int => String
f(c, i, _) : Real => String

f(c, _, _) : Int x Real => String
f(_, i, _) : Char x Real => String
f(_, _, r) : Char x Int => String
\end{verbatim}

\item \textit{Grammar}
\begin{grammar}
<paren-func-app-or-id> ::=
[ <arguments> ] <id-start> ( <arguments> [a-z_]+ )* [ [0-9] ]
[ <arguments> ]

<arguments> ::= `(' [ `\ ' ] <line-expr-or-unders> [ `\ ' ] `)'

\end{grammar}

\end{itemize}

\newpage

\subsubsection{Prefix and Postfix Functions}
\label{subsubsec:prefixpostfix}

\paragraph{Prefix Functions}

\begin{itemize}

\item \textit{Examples}
\begin{verbatim}
the_value:1
non_empty_l:l
error:e
result:r
apply(the_value:_)to_all_in(_)
\end{verbatim}

\item \textit{Description}

Prefix functions are automatically generated from \verb|or_type| definitions
(see "Or Types" in section \ref{subsubsec:tdefs}). They are functions that
convert a value of a particular type to a value that is a case of an
\verb|or_type| and has values of this type inside. For example in the
first example above we have:
\begin{verbatim}
1 : Int
the_value:1 : Possibly(Int)
\end{verbatim}
Where the function \verb|thevalue:_| is automatically generated from the
definition of the \verb|Possibly| type:
\begin{verbatim}
or_type Possibly(T1)
values
  the_value:T1 | no_value
\end{verbatim}
And it has the type \verb|T1 => Possibly(T1)|.
\\\\
These functions are called prefix functions because they are prepended to
their argument. However, they can also be used as arguments to other function
with an underscore in their argument.
An illustration of the aforementioned is the last example, where the function
\verb|the_value:_| is an argument of the function \verb|apply(_)to_all_in(_)|.

\item \textit{Grammar}
\begin{grammar}
<pre-func> ::= <simple-id> `:'

<pre-func-app> ::= <pre-func> <operand>
\end{grammar}

\end{itemize}

\newpage
\paragraph{Postfix Functions}

\begin{itemize}

\item \textit{Examples}
\begin{verbatim}
name.first_name
list.head
date.year
tuple.1st
apply(_.1st)to_all_in(_)
\end{verbatim}

\item \textit{Description}

Postfix functions are automatically generated from \verb|tuple_type|
definitions (see "Tuple Types" in section \ref{subsubsec:tdefs}). They are
functions that take a \verb|tuple_type| value and return a particular field
(i.e. projection functions). For example in the first example above we have:
\begin{verbatim}
name : Name
name.first_name : String
\end{verbatim}
Where the function \verb|_.first_name| is automatically generated from the
definition of the \verb|Name| type:
\begin{verbatim}
tuple_type Name
value (first_name, last_name) : String^2
\end{verbatim}
And it has the type \verb|Name => String|.
\\\\
There are also the following special projection functions that work on all
tuples:
"\verb|_.1st|", "\verb|_.2nd|", "\verb|_.3rd|", "\verb|_.4th|", "\verb|_.5th|"
.
For the 4th example above, assuming:
\begin{verbatim}
tuple : Int x String
\end{verbatim}
We have:
\begin{verbatim}
tuple.1st : Int
\end{verbatim}
The general types of these functions are:
\begin{verbatim}
_.1st : (@A)Is(@B)s_1st --> @B => @A
_.2nd : (@A)Is(@B)s_2nd --> @B => @A
...
\end{verbatim}
These functions are called postfix functions because they are appended to their
argument. However, they can also be used as arguments to other function with an
underscore in their argument.  An illustration of the aforementioned is the
last example, where the function "\verb|_.1st|" is an argument of the function
"\verb|apply(_)to_all_in(_)|".
\\\\
The is a special postfix function called "\verb|_.change|" which is described
in the following paragraph.

\item \textit{Grammar}
\begin{grammar}
<post-func> ::= `.' ( <simple-id> | <special-id> )

<special-id> ::= `1st' | `2nd' | `3rd' | `4th' | `5th'

<post-func-app> ::=
( <basic-expr> | <paren-expr> | `_' )
( <dot-change> | <post-func>+ [ <dot-change> ] )
\end{grammar}

\end{itemize}

\paragraph{The ".change" Function}

\begin{itemize}

\item \textit{Examples}

\begin{verbatim}
state.change{counter = counter + 1}
tuple.change{1st = 42, 3rd = 17}
point.change{x = 1.62, y = 2.72, z = 3.14}
apply(_.change{1st = 1st + 1})to_all_in(_)
x.change{1st = _, 3rd = _}
\end{verbatim}

\item \textit{Description}
\\\\
The "\verb|_.change|" function is a special postfix function that works an all
tuples. It returns a new tuple that is the same as the input tuple except for
some fields that change. Which fields change and to what new value is specified
inside curly brackets after the "\verb|_.change|". The following special
identifiers can be used for referring to the fields:
"\verb|1st|", "\verb|2nd|", "\verb|3rd|", "\verb|4th|", "\verb|5th|"
(2nd, 4th and 5th example). If the tuple is of a tuple type, the identifiers of
the fields specified in the type definition can be used (1st and 3rd example).
Therefore, we are assuming the following (or similar) if the examples are to
type check:

\begin{verbatim}
tuple_type MyStateType
value (..., counter, ...) : ... x Int x ...

state : MyStateType


tuple : Int x SomeType x Int (x ...)


tuple_type Point
value (x, y, z) : Real^3

point : Point


apply(_.change{1st = 1st + 1})to_all_in(_)
  : (@A)And(Int)AddTo(@A), (@A)Is(@B)s_1st --> ListOf(@B)s => ListOf(@B)s


assuming  x : Int x Real x String
x.change{1st = _, 3rd = _} : Int x String => Int x Real x String
\end{verbatim}
The changes of the fields have the following structure:
"\verb|field = <expression of new value>|"
and they are seperated by commas. The input tuple's fields (i.e. the
"old" values) can be used inside the expression of a new value and they are
referred to by the field identifier (1st and 4th example). Underscores can be
used as the expressions of some new values which makes the whole expression
a function that expects those new values as arguments (last example).
\item \textit{Grammar}

\begin{grammar}
<dot-change> ::=
`.change{' [ `\ ' ] <field-change> ( <comma> <field-change> )* [ `\ ' ] `}'

<field-change> ::= ( <simple-id> | <special-id> ) <equals> <line-expr-or-under>

<equals> ::= [ `\ ' ] `=' [ `\ ' ]
\end{grammar}

\end{itemize}

\subsection{Operators}

\subsubsection{Function Application and Function Composition Operators}

\paragraph{Function Application Operators}

\begin{center}
\begin{tabular}{ |c|c|c| }
\hline
Operator & Type
\\
\hline
\hline
\verb|->| & \verb| T1 x (T1 => T2) => T2|
\\
\hline
\verb|<-| & \verb| (T1 => T2) x T1 => T2|
\\
\hline
\end{tabular}
\end{center}
The function application operators "\verb|->|" and "\verb|<-|" are a
different way to apply functions to arguments than the usual parenthesis
function application.  They are meant to look like arrows that point from the
argument to the function.  These operators are very useful for chaining many
function applications without the clutter of having to open and close
parentheses for each one of the functions.  For example, assuming we have the
following functions with the behaviour suggested by their names and types:
\begin{verbatim}
apply(_)to_all_in(_) : (T1 => T2) x ListOf(T1)s => ListOf(T2)s
str_len(_) : String => Int
filter(_)with(_) : ListOf(T1)s x (T1 => Bool) => ListOf(T1)s
(_)is_odd : Int => Bool
sum_ints(_) : ListOf(Int)s => Int
\end{verbatim}
And a list of strings:
\begin{verbatim}
strings : ListOf(String)s
\end{verbatim}
Here is a simple way to get the total number of characters in all the strings
that have odd length:
\begin{verbatim}
chars_in_odd_length_strings : Int
  = apply(str_len(_))to_all_in(strings) -> filter(_)with((_)is_odd) -> sum_ints(_)
\end{verbatim}
This can be done equivalently using the other operator:
\begin{verbatim}
chars_in_odd_length_strings : Int
  = sum_ints(_) <- filter(_)with((_)is_odd) <- apply(str_len(_))to_all_in(strings)

\end{verbatim}

\newpage

\paragraph{Function Composition Operators}
\begin{center}
\begin{tabular}{ |c|c|c| }
\hline
Operator & Type
\\
\hline
\hline
\verb|o>| & \verb|(T1 => T2) x (T2 => T3) => (T1 => T3)|
\\
\hline
\verb|<o| & \verb|(T2 => T3) x (T1 => T2) => (T1 => T3)|
\\
\hline
\end{tabular}
\end{center}
The function composition operators "\verb|o>|" and "\verb|<o|" are used to
compose functions, each one in the corresponding direction. The use of the
letter '\verb|o|' is meant to be similar to the mathematical function
composition symbol '\(\circ\)' and the symbols '\verb|>|', '\verb|<|' are used
so that the operator points from the function which is applied first to the
function which is applied second.  A neat example using function composition is
the following. Assuming we have the following functions with the behaviour
suggested by their names and types:
\begin{verbatim}
split(_)to_words : String => ListOf(String)s
apply(_)to_all_in(_) : (T1 => T2) x ListOf(T1)s => ListOf(T2)s
reverse_str(_) : String => String
merge_words(_) : ListOf(String)s => String
\end{verbatim}
We can reverse the all the words in a string like so:
\begin{verbatim}
reverse_words_in(_) : String => String
  = split(_)to_words o> apply(reverse_str(_))to_all_in(_) o> merge_words(_)
\end{verbatim}
This can be done equivalently using the other operator:
\begin{verbatim}
reverse_words_in(_) : String => String
  = merge_words(_) <o apply(reverse_str(_))to_all_in(_) <o split(_)to_words
\end{verbatim}

\newpage

\subsubsection{Arithmetic, Comparison and Boolean Operators}

\paragraph{Arithmetic Operators}
\begin{center}
\begin{tabular}{ |c|c|c| }
\hline
Operator & Type
\\
\hline
\hline
\verb|^| & \verb|(@A)To_The(@B)Is(@C) --> @A x @B => @C|
\\
\hline
\verb|*| & \verb|(@A)And(@B)Multiply_To(@C) --> @A x @B => @C|
\\
\hline
\verb|/| & \verb|(@A)Divided_By(@B)Is(@C) --> @A x @B => @C|
\\
\hline
\verb|+| & \verb|(@A)And(@B)Add_To(@C) --> @A x @B => @C|
\\
\hline
\verb|-| & \verb|(@A)Minus(@B)Is(@C) --> @A x @B => @C|
\\
\hline
\end{tabular}
\end{center}
The usual arithmetic operators work as they are expected, similarly to
mathematics and other programming languages for the usual types. However, they
are generalized. The examples below show their generality:
\begin{verbatim}
>> 1 + 1
  : Int
  ==> 2
>> 1 + 3.14
  : Real
  ==> 4.14
>> 'a' + 'b'
  : String
  ==> "ab"
>> 'w' + "ord"
  : String
  ==> "word"
>> "Hello " + "World!"
  : String
  ==> "Hello World!"
>> 5 * 'a'
  : String
  ==> "aaaaa"
>> 5 * "hi"
  : String
  ==> "hihihihihi"
>> "1,2,3" - ','
  : String
  ==> "123"
\end{verbatim}
Let's analyze further the example of addition. The type can be read as such:
the '\verb|+|' operator has the type \\ \verb|@A x @B => @C|, provided that the
type proposition \verb|(@A)And(@B)Add_To(@C)| holds. This proposition being
true, means that addition has been defined for these three types (see section
"Type Logic" \ref{subsec:typelogic} for more on type propositions). For
example, by the examples above we can deduce that the following propositions
are true (in the order of the examples):
\begin{verbatim}
(Int)And(Int)Add_To(Int)
(Int)And(Real)Add_To(Real)
(Char)And(Char)Add_To(String)
(Char)And(String)Add_To(String)
(Int)And(Char)Multiply_To(String)
(Int)And(String)Multiply_To(String)
(String)Minus(Char)Is(String)
\end{verbatim}
This allows us to use the familiar arithmetic operators in types that are not
necessarily numbers but it is somewhat intuitively obvious what they should do
in those other types. Furthermore, their behaviour can be defined by the user
for new user defined types!

\paragraph{Comparison, Boolean and Bitwise Operators}

\begin{center}
\begin{tabular}{ |c|c|c| }
\hline
Operator & Type
\\
\hline
\hline
\verb|==| & \verb|(@A)And(@B)Can_Be_Equal --> @A x @B => Bool|
\\
\hline
\verb|!=| & \verb|(@A)And(@B)Can_Be_Unequal --> @A x @B => Bool|
\\
\hline
\verb|>| & \verb|(@A)Can_Be_Greater_Than(@B) --> @A x @B => Bool|
\\
\hline
\verb|<| & \verb|(@A)Can_Be_Less_Than(@B) --> @A x @B => Bool|
\\
\hline
\verb|>=| & \verb|(@A)Can_Be_Gr_Or_Eq_To(@B) --> @A x @B => Bool|
\\
\hline
\verb|<=| & \verb|(@A)Can_Be_Le_Or_Eq_To(@B) --> @A x @B => Bool|
\\
\hline
\verb|&| & \verb|(@A)Has_And --> @A^2 => @A|
\\
\hline
\texttt{|} & \verb|(@A)Has_Or --> @A^2 => @A|
\\
\hline
\end{tabular}
\end{center}
Comparison operators are also generalized. The main reason for the
generalization is to be able to compare numbers of different types. Consider
the following example:

\begin{verbatim}
>> 1
  : Int
  ==> 1
>> 1.1
  : Real
  ==> 1.1
>> 1.1 == 1
  : Bool
  ==> false
>> 1.0 == 1
  : Bool
  ==> true
\end{verbatim}
In order for the example to work we need to be able to compare integers and
reals.  Similarly, all the comparison operators need to be able to work on
arguments of different types.
\\\\
Boolean "and" and bitwise "and" are combined into one general "and" operator
(\verb|&|). The same applies to the "or" operator (\texttt{|}).

\newpage

\subsubsection{Environment Action Operators}
\label{subsec:envacts}

\begin{center}
\begin{tabular}{ |c|c|c| }
\hline
Operator & Type
\\
\hline
\hline
\verb|;>| & \verb|(@E)Has_Use --> @E(T1) x (T1 => @E(T2)) => @E(T2)|
\\
\hline
\verb|;| & \verb|(@E)Has_Then --> @E(T1) x @E(T2) => @E(T2)|
\\
\hline
\end{tabular}
\end{center}

\paragraph{Simple Example Program}
\begin{verbatim}
main: (EmptyVal)WithIO
  = print_string("I'll repeat the line") ; get_line ;> print_string(_)
\end{verbatim}
The example above demonstrates the use of the environment action operators with
the \verb|WithIO| environment type, which is how IO is done in lcases. Some
light can be shed on how this is done, if we take a look at the types (as
always!):
\begin{verbatim}
print_string(_): String => (EmptyVal)WithIO
print_string("I'll repeat the line"): (EmptyVal)WithIO

get_line: (String)WithIO

# for the "then" operator we have:
; : (@E)Has_Then --> @E(T1) x @E(T2) => @E(T2)

# therefore in the following expression:
#     print_string("I'll repeat the line") ; get_line
# the only way to match the types is if @E = WithIO, T1 = EmptyVal, T2 = String
# from which it follows that (WithIO)Has_Then
# and the "then" operator (in this particular case) has type
#     (EmptyVal)WithIO x (String)WithIO => (String)WithIO
# for the whole expression we have:
print_string("I'll repeat the line") ; get_line
  : (String)WithIO

print_string: String => (EmptyVal)WithIO

# for the "use" operator we have:
;> : (@E)Has_Use --> @E(T1) x (T1 => @E(T2)) => @E(T2)

# therefore in the following expression:
#     print_string("I'll repeat the line") ; get_line ;> print_string(_)
# the only way to match the types is if @E = WithIO, T1 = String, T2 = EmptyVal
# from which it follows that (WithIO)Has_Use
# and the "use" operator (in this particular case) has type
#     (String)WithIO x (String => (EmptyVal)WithIO) => (EmptyVal)WithIO
# for the whole expression we have:
print_string("I'll repeat the line") ; get_line ;> print_string(_)
  : (EmptyVal)WithIO
\end{verbatim}

\newpage
\paragraph{Another Example Program}
\begin{verbatim}
main: (EmptyVal)WithIO
  = print_string <- "Hello! What's your name?" ; get_line ;> name =>
    print_string("Nice to meet you " + name + "!")

print_string(_): String => (EmptyVal)WithIO
"Hello! What's your name?": String
<- : (T1 => T2) x T1 => T2

# matching the types: T1 = String, T2 = (EmptyVal)WithIO
print_string <- "Hello! What's your name?"
  : (EmptyVal)WithIO

# Similarly to the previous example program
print_string <- "Hello! What's your name?" ; get_line
  : (String)WithIO

name => print_string("Nice to meet you " + name + "!")
  : String => (EmptyVal)WithIO

# Here we have a function expression as the second operand
# of the ";>" operator:
#   print_string <- "Hello! What's your name?" ; get_line ;> name =>
#   print_string("Nice to meet you " + name + "!")
# Whenever this happens the function extends to the end of the expression.
# So the expression:
#   name => print_string("Nice to meet you " + name + "!")
# is the second operand.
# This is important to note as it could be longer and have other subexpressions
# in it. For example it could be:
#   name => print_string("Nice to...) ; print_string("How old are you?") ;
#   get_line ;> age => print_string(...) ; ...
# Here again the whole expression from "name => ..." till the end would be the
# second operand to the ";>" operator (and as you can see it has more ";>"
# operators and operands of those operators inside)
#
# Putting all together (similarly to the previous example program):
print_string <- "Hello! What's your name?" ; get_line ;> name =>
print_string("Nice to meet you " + name + "!")
  : (EmptyVal)WithIO
\end{verbatim}

\paragraph{Description}\pend
The environment action operators are used to combine values that do environment
actions into values that do more complicated environment actions. Environment
actions are type functions that take a type argument and produce a type (just
like ListOf(_)s). These type functions have the "then" operator (\verb|;|)
and the "use" operator (\verb|;>|) defined for them.  A value of the type
\verb|@E(T1)| where \verb|(@E)Has_Then| does an environment action of type
\verb|@E| that produces a value of type \verb|T1| which can then be
combined with another one with the "then" operator. Similarly, with the "use"
operator the produced value of an action can be used by a function that returns
another action.
\\\\
The effect of the "\verb|;|" operator described in words is the following:
given a value of type \verb|@E(T1)| and a value of type \verb|@E(T2)|
(which are environment actions that produce values of type \verb|T1| and
\verb|T2| respectively), create a new value the does both actions (provided
the first did not result in an error).  The overall effect is a value which is
an environment action of type \verb|@E| (the combination of the "smaller"
actions) which produces a value of type \verb|T2| (the one produced by the
second action) and therefore it is of type \verb|@E(T2)|.
\\\\
Note that the value of type \verb|T1| produced by the first action is not
used anywhere. This happens mostly when \verb|T1 = EmptyVal| and it is
because values of type \verb|@E(EmptyVal)| are used for their environment
action only \\(e.g. \verb|print_string(...): (EmptyVal)WithIO|).
\\\\
How the two environment actions of the \verb|@E(T1)| and \verb|@E(T2)|
values are combined to produce the new environment action is specific to the
environment action type \verb|@E|.
\\\\
The effect of the "\verb|;>|" operator described in words is the following:
given a value of type \verb|@E(T1)| (which is an environment action of type
\verb|@E| that produces a value of type \verb|T1|) and a value of type
\verb|T1 => @E(T2)| (which is a function that takes a value of type
\verb|T1| and returns an environment action of type \verb|@E| that produces
a value of type \verb|T2|), combine those two values by creating a value that
does the following:
\begin{itemize}
\item
Performs the first action that produces a value of type \verb|T1|

\item
Takes the value of type \verb|T1| produced (provided there was no error) and
passes it to the function of type \verb|T1 => @E(T2)| that then returns an
action

\item
Perfoms the resulting action
\end{itemize}
The overall effect is an environment action  of type \verb|@E| that produces
a value is of type \verb|T2| and therefore the new value is of type
\verb|@E(T2)|.

\newpage

\subsubsection{Operator Expressions}

\begin{itemize}
\item \textit{Examples}
\begin{verbatim}
1 + 2
1 + x * 3^y
"Hello " + "World!"
x -> f -> g
f o> g o> h
x == y
x >= y - z & x < 2 * y
get_line ; get_line ;> line => print("Second line: " + line)
2 * _
_ - 1
"Hello " + "it's me, " + _
"Hi, I am " + _ + " and I am " + _ + " years old"
\end{verbatim}

\item \textit{Description}

Operator expressions are expressions that use operators. Operators act like
two-argument-functions that are placed in between their arguments (operands).
Therefore, they have function types and they act as it is described in their
respective sections above this one.
\\\\
An operator expression might have multiple operators. The order of operations
is explained in the next section ("Complete Table, Precedence and
Associativity") in Table \ref{table:precassoc}.
\\\\
Just like functions, the operands of an operator, must have types that match
the types expected by the operator.
\\\\
It is possible for the second operand of an operator to be a function
expression.  This is mostly useful with the "\verb|;>|" operator (see
previous section: "Environment Operators"), but it is also possible with the
following operators: "\verb|->|", "\verb|o>|", "\verb|<o|".
\\\\
It is possible to use an underscore as an operand. An operator expression with
underscore operands becomes a function that expects those operands as
arguments.  This is best demonstrated by the types of the last four examples:

\begin{verbatim}
2 * _ : Int => Int
_ - 1 : Int => Int
"Hello " + "it's me, " + _ : String => String
"Hi, I am " + _ + " and I am " + _ + " years old" : String^2 => String
\end{verbatim}
Note: These are not the most general types for the examples but they are
compatible and good enough for their illustration purposes.

\newpage

\item \textit{Big Operator Expressions}\\\\
\textbf{Example}

\begin{verbatim}
"Hello, I'm a big string that's going to contain multiple values from " +
"inside the imaginary program that I'm a part of. Here they are:\n" +
"value1 = " + value1 + ", value2 = " + value2 + ", value3 = " + value3 +
", value4 = " + value4 + ", value5 = " + value5
\end{verbatim}

\textbf{Description}

It is possible to stretch a (big) operator expression over multiple lines.
In that case:
\begin{itemize}
\item
The operator expression must split in a new line after an operator (not an
argument).

\item
Every line after the first must be indented so that in begins at the column
where the first line of the operator expression begun.

\item
The precise indentation rules are described in the section
"Indentation System" \ref{subsubsec:indsys}.
\end{itemize}

\item \textit{Grammar}
\begin{grammar}
<op-expr> ::= <line-op-expr> | <big-op-expr>
\\

<op-expr-start> ::= ( <operand> <op> )+

<line-op-expr> ::= <op-expr-start> ( <operand> | <line-func-expr> )
\\

<big-op-expr> ::=
<big-op-expr-op-split> | <big-op-expr-func-split>
\\

<big-op-expr-op-split> ::=
<op-split-line>+ [ <op-expr-start> ] ( <operand> | <func-expr> )

<op-split-line> ::=
( <op-expr-start> ( <nl> | <oper-fco> ) | <oper-fco> ) <indent>

<oper-fco> ::= <operand> `\ ' <func-comp-op> `\\n'
\\

<big-op-expr-func-split> ::=
<op-expr-start> ( <big-func-expr> | <cases-func-expr> )
\\

<operand> ::= <basic-or-app-expr> | <paren-expr> | `_'
\\

<op> ::= `\ ' <func-comp-op> `\ ' | [ `\ ' ] <optional-spaces-op> [ `\ ' ]

<func-comp-op> ::= `o>' | `<o'

<optional-spaces-op> ::=
`->' | `<-' | `^' | `*' | `/' | `+' | `-' | `==' | `!=' | `>' | `<' | `>=' |
`<=' | `\&' | `|' | `;>' | `;'
\end{grammar}
\end{itemize}

\newpage
\subsubsection{Complete Table, Precedence and Associativity}

\begin{table}[h]

\caption{
The complete table of lcases operators along with their types and
their short descriptions.
}

\begin{center}
\begin{tabular}{ |c|c|c| }
\hline
Operator & Type & Description
\\
\hline
\hline
\verb|->| & \verb|T1 x (T1 => T2) => T2| & Right Function Application
\\
\hline
\verb|<-| & \verb|(T1 => T2) x T1 => T2| & Left Function Application
\\
\hline
\verb|o>| & \verb|(T1 => T2) x (T2 => T3) => (T1 => T3)| &
Right Function Composition
\\
\hline
\verb|<o| & \verb|(T2 => T3) x (T1 => T2) => (T1 => T3)| &
Left Function Composition
\\
\hline
\verb|^| & \verb|(@A)To_The(@B)Is(@C) --> @A x @B => @C| &
General Exponentiation
\\
\hline
\verb|*| & \verb|(@A)And(@B)Multiply_To(@C) --> @A x @B => @C| &
General Multiplication
\\
\hline
\verb|/| & \verb|(@A)Divided_By(@B)Is(@C) --> @A x @B => @C| &
General Division
\\
\hline
\verb|+| & \verb|(@A)And(@B)Add_To(@C) --> @A x @B => @C| &
General Addition
\\
\hline
\verb|-| & \verb|(@A)Minus(@B)Is(@C) --> @A x @B => @C| &
General Subtraction
\\
\hline
\verb|==| & \verb|(@A)And(@B)Can_Be_Equal --> @A x @B => Bool| &
General Equality
\\
\hline
\verb|!=| & \verb|(@A)And(@B)Can_Be_Unequal --> @A x @B => Bool| &
General Inequality
\\
\hline
\verb|>| & \verb|(@A)Can_Be_Greater_Than(@B) --> @A x @B => Bool| &
General Greater Than
\\
\hline
\verb|<| & \verb|(@A)Can_Be_Less_Than(@B) --> @A x @B => Bool| &
General Less Than
\\
\hline
\verb|>=| & \verb|(@A)Can_Be_Gr_Or_Eq_To(@B) --> @A x @B => Bool| &
General Greater Than or Equal To
\\
\hline
\verb|<=| & \verb|(@A)Can_Be_Le_Or_Eq_To(@B) --> @A x @B => Bool| &
General Less Than or Equal To
\\
\hline
\verb|&| & \verb|(@A)Has_And --> @A^2 => @A| & General And
\\
\hline
\texttt{|} & \verb|(@A)Has_Or --> @A^2 => @A| & General Or
\\
\hline
\verb|;>| & \verb|(@E)Has_Use --> @E(T1) x (T1 => @E(T2)) => @E(T2)| &
"Use" Environment Action
\\
\hline
\verb|;| & \verb|(@E)Has_Then --> @E(T1) x @E(T2) => @E(T2)| &
"Then" Environment Action
\\
\hline
\end{tabular}
\end{center}

\label{table:allops}

\end{table}
\newpage\noindent
The order of operations is done from highest to lowest precedence. In the same
level of precedence the order is done from left to right if the associativity
is "Left" and from right to left if the associativity is "Right". For the
operators that have associativity "None" it is not allowed to place them in the
same operator expression. The precedence and assosiativity of the operators
is shown in the table below.

\begin{table}[h]

\caption{ The table of precedence and associativity of the lcases operators.  }

\begin{center}
\begin{tabular}{ |c|c|c| }
\hline
Operator & Precedence & Associativity
\\
\hline
\hline
\verb|->| & 10 (highest) & Left
\\
\hline
\verb|<-| & 9 & Right
\\
\hline
\verb|o> <o| & 8 & Left
\\
\hline
\verb|^| & 7 & Right
\\
\hline
\verb|* /| & 6 & Left
\\
\hline
\verb|+ -| & 5 & Left
\\
\hline
\verb|== != > < >= <=| & 4 & None
\\
\hline
\verb|&| & 3 & Left
\\
\hline
\texttt{|} & 2 & Left
\\
\hline
\verb|;> ;| & 1 & Left
\\
\hline
\end{tabular}
\end{center}

\label{table:precassoc}

\end{table}

\newpage
\subsection{Function Expressions}
\label{subsec:funcexprs}

Function expressions are divided into \textbf{regular function expressions} and
\textbf{"cases" function expressions} which are described in the following
sections.
\begin{grammar}
<func-expr> ::= <line-func-expr> | <big-func-expr> | <cases-func-expr>
\end{grammar}

\subsubsection{Regular Function Expressions}

\begin{itemize}
\item \textit{Examples}

\begin{verbatim}
a => 17 * a + 42
(a, b) => a + 2*b
(x, y, z) => sqrt(x^2 + y^2 + z^2)
* => 42
(x, *, z) => x + z
((x1, y1), (x2, y2)) => (x1 + x2, y1 + y2)
\end{verbatim}

\item \textit{Description}

Regular function expressions are used to define functions or be part of bigger
expressions as anonymous functions. They are comprised by their parameters and
their body.
\\\\
Parameters have identifiers. There is either only one parameter, in which case
there is no parenthesis, or there are many, in which case they are in
parentheses, seperated by commas. If a parameter is not needed it can be left
empty by having an asterisk instead of an identifier (3rd and 4th example).
If a parameter is a tuple itself it can be matched further by using parentheses
and giving identifiers to its fields (5th example).
\\\\
The parameters and the body are seperated by the funciton arrow
("\verb|=>|"). The body is an operator or basic expression.

\item \textit{Big Function Expressions}\\\\
\textbf{Example}

\begin{verbatim}
(value1, value2, value3, value4, value5, value6, value7) =>
print_line("value1 = " + value1 + ", value2 = " + value2 + ", value3 = " + value3) ;
print_line("value4 = " + value4 + ", value5 = " + value5 + ", value6 = " + value6) ;
print_line("value7 = " + value7)
\end{verbatim}

\textbf{Description}

It is possible to stretch a (big) function expression over multiple lines.
In that case:
\begin{itemize}
\item
The function expression must split in a new line after the function arrow
("\verb|=>|").

\item
Every line after the first must be indented so that in begins at the column
where the first character of the parameters was in the first line.

\item
The precise indentation rules are described in the section
"Indentation System" \ref{subsubsec:indsys}.
\end{itemize}

\item \textit{Grammar}
\begin{grammar}
<line-func-expr> ::= <parameters> [ `\ ' ] `=>' <line-func-body>

<big-func-expr> ::= <parameters> [ `\ ' ] `=>' <big-func-body>

<parameters> ::=
<identifier> | `*' |
`(' [ `\ ' ] <parameters> ( <comma> <parameters> )+ [ `\ ' ] `)'

<line-func-body> ::=
[ `\ ' ]
(
<basic-or-app-expr> | <line-op-expr> |
`(' [ `\ ' ] <line-func-expr> [ `\ ' ] `)'
)

<big-func-body> ::=
<nl> <indent>
(
<basic-or-app-expr> | <op-expr> | `(' [ `\ ' ] <line-func-expr> [ `\ ' ] `)'
)
\end{grammar}
\end{itemize}

\subsubsection{"cases" Function Expressions}
\label{subsubsec:casessyntax}

\begin{itemize}
\item \textit{Examples}

\begin{verbatim}
print_sentimental_bool(_): Bool => IO
  = cases
    true => print <- "It's true!! :)"
    false => print <- "It's false... :("

or_type TrafficLight
values green | amber | red

print_sentimental_traffic_light(_): TrafficLight => IO
  = cases
    green => print <- "It's green! Let's go!!! :)"
    amber => print <- "Go go go, fast!"
    red => print <- "Stop right now! You're going to kill us!!"

(_)is_not_red: TrafficLight => Bool
  = cases
    green => true
    amber => true
    red => false

(_)is_seventeen_or_forty_two: Int => Bool
  = cases
    17 => true
    42 => true
    ... => false

traffic_lights_match(_, _): TrafficLight^2 => Bool
  = (cases, cases)
    (green, green) => true
    (amber, amber) => true
    (red, red) => true
    ... => false

gcd_of(_)and(_): Int^2 => Int
  = (x, cases)
    0 => x
    y => gcd_of(y)and((x)mod(y))

(_)is_empty: ListOf(T1)s => Bool
  = cases
    empty_l => true
    non_empty_l:* => false

apply(_)to_all_in(_): (T1 => T2) x ListOf(T1)s => ListOf(T2)s
  = (f, cases)
    empty_l => empty_l
    non_empty_l:list => non_empty_l:(f <- list.head, apply(f)to_all_in(list.tail))

\end{verbatim}

\newpage

\item \textit{Description}

"\verb|cases|" is a keyword that works as a special parameter.  Instead of
giving the name "\verb|cases|" to that parameter, it is used to pattern match
on the possible values of that parameter and return a different result for each
particular case.
\\\\
The last case can be "\verb|... => (body of default case)|" to capture all
remaining cases while dismissing the value (e.g.
"\verb|is_seventeen_or_forty_two|" example), or it can be
"\verb|some_id => (body of default case)|" to capture all remaining
cases while being able to use the value with the name "\verb|some_id|"
(e.g.  "\verb|y|" in "\verb|gcd|" example).
\\\\
It is possible to use the "\verb|cases|" keyword in multiple parameters to match on
all of them. By doing that, each case represents a particular combination of
values for the parameters involved\\(e.g. \verb|traffic_lights_match|
example).
\\\\
It is also possible to use a "\verb|where|" expression below a particular case.
The "\verb|where|" expression must be indented two spaces more than than the
line where that particular case begins.
\\\\
A function expression that uses the "\verb|cases|" syntax must contain the
"\verb|cases|" keyword in at least one parameter. The number of matching
expressions in all cases must be the same as the number of parameters with the
"\verb|cases|" keyword.

\item \textit{Grammar}
\begin{grammar}
<cases-func-expr> ::= <cases-params> <case>+ [ <end-case> ]

<cases-params> ::=
""\\
<identifier> | `cases' | `*' |
`(' [ `\ ' ] <cases-params> ( <comma> <cases-params> )+ [ `\ ' ] `)'
\\

<case> ::=  <nl> <indent> <outer-matching> [ `\ ' ] `=>' <case-body>

<end-case> ::=
<nl> <indent> ( `...' | <identifier> ) [ `\ ' ] `=>' <case-body>
\\

<outer-matching> ::= <simple-id> | <matching>

<matching> ::=
<literal> | <pre-func> <inner-matching> | <tuple-matching> | <list-matching>

<inner-matching> ::= `*' | <identifier> | <matching>

<tuple-matching> ::=
`(' [ `\ ' ] <inner-matching> ( <comma> <inner-matching> )+ [ `\ ' ] `)'

<list-matching> ::= ""\\
`[' [ `\ ' ]
[ <inner-matching> ( <comma> <inner-matching> )* [ <rest-list-matching> ] ]
[ `\ ' ] `]'

<rest-list-matching> ::= <comma> [ <simple-id> <equals> ] `...'
\\

<case-body> ::= <line-func-body> | <big-func-body> [ <where-expr> ]
\end{grammar}

\end{itemize}

\newpage

\subsection{Value Definitions and "where" Expressions}
\label{subsec:valdefswhere}

\subsubsection{Value Definitions}
\label{subsubsec:valdefs}

\begin{itemize}

\item \textit{Examples}

\begin{verbatim}
foo: Int
  = 42

f(_, _, _): Int^3 => Int
  = (a, b, c) => a + b * c

val1, val2, val3: Int, Bool, Char
  = 42, true, 'a'

int1, int2, int3: all Int
  = 1, 2, 3
\end{verbatim}

\item \textit{Description}

Value definitions are the main building block of lcases programs. To define a
new value you give it a name, a type and an expression. The name is an
identifier which is followed by the "has type" symbol ('\verb|:|') and the
expression of the type of the value. The line below is indented two spaces and
begins with the equal sign and continues with the expression of the value
(which extends to as many lines as needed).
\\\\
A value definition begins either in the first column, where it can be "seen" by
all other value definitions, or it is inside a "\verb|where|" expression (see
section below), where it can be "seen" by the expression above the
"\verb|where|" and all the other definitions in the same "\verb|where|"
expression.
\\\\
A value definition can be followed by a "\verb|where|" expression where
intermediate values used in the value expression are defined. In that case, the
"\verb|where|" expression must be indented two spaces more than the "\verb|=|"
line of the value definition.
\\\\
It is possible to group value definitions together by seperating the names, the
types and the expressions with commas. This is very useful for not cluttering
the program with many definitions for values with small expressions (e.g.
constants).  When grouping definitions together it is also possible to use the
keyword "\verb|all|" to give the same type to all the values.

\item \textit{Grammar}
\begin{grammar}
<value-def> ::= ""\\
<indent> <identifier>
( [ `\ ' ] `:' [ `\ ' ] | <nl> <indent> `:\ ' ) <type>  \\
<nl> <indent> `=\ ' <value-expr> [ <where-expr> ]

<value-expr> ::=
<basic-or-app-expr> | <op-expr> | <func-expr> | <big-tuple> | <big-list>
\\

<grouped-value-defs> ::= ""\\
<indent> <identifier> ( <comma> <identifier> )+ \\
( [ `\ ' ] `:' [ `\ ' ] | <nl> <indent> `:\ ' )
( <type> ( <comma> <type> )+ | `all\ ' <type> ) \\
<nl> <indent> `=\ ' <line-exprs> ( <nl> <indent> <comma> <line-exprs> )*

<line-exprs> ::= <line-expr> ( <comma> <line-expr> )*
\end{grammar}

\end{itemize}

\newpage

\subsubsection{"where" Expressions}
\label{subsubsec:whereexprs}

\begin{itemize}

\item \textit{Examples}
\begin{verbatim}
sort(_): ListOf(Int)s => ListOf(Int)s
  = cases
    empty_l => empty_l
    non_empty_l:l => sort(less_l) + l.head + sort(greater_l)
      where
      less_l, greater_l: all ListOf(Int)s
        = filter(l.tail)with(_ < l.head), filter(l.tail)with(_ >= l.head)

sum_nodes(_): TreeOf(Int)s => Int
  = tree =>
    tree.root + apply(sum_nodes(_))to_all_in(tree.subtrees) -> sum_list(_)
    where
    sum_list(_) : ListOf(Int)s => Int
      = cases
        empty_l => 0
        non_empty_l:l => l.head + sum_list(l.tail)

big_string : String
  = s1 + s2 + s3 + s4
    where
    s1, s2, s3, s4 : all String
      = "Hello, my name is Struggling Programmer."
      , " I have tried way too many times to fit a big chunk of text"
      , " inside my program, without it hitting the half-screen mark!"
      , " I am so glad I finally discovered lcases!"
\end{verbatim}

\item \textit{Description}

"\verb|where|" expressions allow the programmer to use values inside an
expression and define them below it. They are very useful for reusing or
abbreviating expressions that are specific to a particular definition or case.
\\\\
A "\verb|where|" expression begins by a line that only has the word
"\verb|where|" in it. It is indented as described in the "Value Definitions"
(\ref{subsubsec:valdefs}) or "'cases' Function Expressions"
(\ref{subsubsec:casessyntax}) sections.  The definitions are placed below the
"\verb|where|" line and must have the same indentation.

\item \textit{Grammar}
\begin{grammar}
<where-expr> ::=
<nl> <indent> `where'
<nl> <value-def-or-defs> ( <nl> <nl> <value-def-or-defs> )*

<value-def-or-defs> ::= <value-def> | <grouped-value-defs>
\end{grammar}

\end{itemize}

\newpage

\section{Language Description: Types and Type Logic}

\subsection{Types}
\label{subsec:types}

The constructs regarding types are \textbf{type expressions}, \textbf{type
definitions} and \textbf{type nicknames} and they are described in the
following sections.

\subsubsection{Type Expressions}

Type expressions are divided into the following categories:
\begin{itemize}
\item Type Identifiers
\item Type Variables
\item Type Application Types
\item Product Types
\item Function Types
\item Conditional Types
\end{itemize}
which are described in the following paragraphs.
\\\\
The grammar of a type expression is:
\begin{grammar}
<type> ::= [ <condition> ]  <simple-type>

<simple-type> ::=
<param-t-var> | <type-app-id-or-ahtv> | <power-type> | <prod-type> |
<func-type>
\end{grammar}

\paragraph{Type Identifiers}

\begin{itemize}
\item \textit{Examples}
\begin{verbatim}
Int     Real     Char     String     SelfReferencingType
\end{verbatim}

\item \textit{Description}

A type identifier is either the name of a basic type (Int, Real, Char, String)
or the name of some defined type that has no type parameters. It begins with a
capital letter and is followed by capital or lowercase letters.

\item \textit{Grammar}
\begin{grammar}
<type-id> ::= [A-Z] [A-Za-z]*
\end{grammar}
\end{itemize}

\paragraph{Type Variables}\pend
Type Variables are placeholders inside bigger type expressions that can be
substituted with various types. This makes the bigger type expression an
expression of a \textbf{polymorphic} type. The types of polymorphism that exist
in lcases are \textbf{parametric polymorphism} and \textbf{ad hoc
polymorphism}. Type variables for each of the two types have different syntax
and they are described in the following paragraphs.

\subparagraph{Parametric Type Variables}

\begin{itemize}
\item \textit{Examples}
\begin{verbatim}
T1     T2     T3
\end{verbatim}

\item
\textit{Examples of parametric type variables inside bigger type expressions}
\begin{verbatim}
T1 => T1
(T1 => T2) x (T2 => T3) => (T1 => T3)
(T1^2 => T1) x T1 x ListOf(T1)s => T1
\end{verbatim}

\item \textit{Description}

Parametric type variables can be substituted with any type and the program will
type check.  The simplest example of a polymorphic type with a parametric type
variable is the type of the identity function where we have:
\begin{verbatim}
id(_): T1 => T1
  = x => x

id(1): Int
  where T1 is substituted by Int and id gets the type Int => Int

id("Hello"): String
  where T1 is substituted by String and id gets the type String => String
\end{verbatim}

A parametric type variable is written with capital "T" followed by a digit.

\item \textit{Grammar}
\begin{grammar}
<param-t-var> ::= `T' [0-9] \\
\end{grammar}
\end{itemize}

\subparagraph{Ad Hoc Type Variables}

\begin{itemize}
\item \textit{Examples}
\begin{verbatim}
@A @B @C @T
\end{verbatim}

\item \textit{Examples of ad hoc type variables inside bigger type expressions}
\begin{verbatim}
(@T)Has_Str_Rep --> @T => String
(@A)Is(@B)s_First --> @B => @A
(@A)And(@B)Can_Be_Equal --> @A x @B => Bool
(@A)And(@B)Add_To(@C) --> @A x @B => @C
\end{verbatim}

\item \textit{Description}

Ad hoc type variables are like parametric type variables with the difference
that any type by which they are substituted must satisfy certain conditions in
order for the program to type check. These conditions come in the form of type
propositions (see Type Logic section \ref{subsec:typelogic}). Therefore, any ad
hoc type variable must also appear in the condition as shown in the examples.
\\\\
An ad hoc type variable is written with an '@' followed by any capital letter.

\item \textit{Grammar}
\begin{grammar}
<ad-hoc-t-var> ::= `@' [A-Z]
\end{grammar}
\end{itemize}

\paragraph{Type Application Types}

\begin{itemize}
\item \textit{Examples}
\begin{verbatim}
Possibly(Int)
ListOf(Real)s
TreeOf(String)s
Result(Int)OrError(String)
ListOf(Int => Int)s
ListOf(T1)s
\end{verbatim}

\item \textit{Description}

Type application types are types that are produced by passing arguments
to a type function generated by a \verb|tuple_type| or an \verb|or_type|
definition. For example, given the definition of \verb|ListOf(T1)s|:
\begin{verbatim}
or_type ListOf(T1)s
values non_empty_l:NonEmptyListOf(T1)s | empty_l
\end{verbatim}
We have that \verb|ListOf(_)s| is a type function that receives one type
parameter and returns a resulting type. For example \verb|ListOf(Int)s| is
the result of passing the type argument \verb|Int| to \verb|ListOf(_)s|.
\\\\
Type application types have the same form as the name in the
\verb|tuple_type| or \verb|or_type| definition, with the difference that
type parameters are substituted by the expressions of the type arguments.

\item \textit{Grammar}
\begin{grammar}
<type-app-id-or-ahtv> ::=
[ <types-in-paren> ] <taioa-middle> [ <types-in-paren> ]

<taioa-middle> ::=
<type-id> ( <types-in-paren> [A-Za-z]+ )* | <ad-hoc-t-var>

<types-in-paren> ::=
`(' [ `\ ' ] <simple-type> ( <comma> <simple-type> )* [ `\ ' ] `)'
\end{grammar}
\end{itemize}

\newpage

\paragraph{Product Types}

\begin{itemize}
\item \textit{Examples}
\begin{verbatim}
Int x Real x String
ListOf(Int)s x Int x ListOf(String)s
(Int => Int) x (Int x Real) x (Real => String)
Int^2 x Int^2
Real^3 x Real^3
\end{verbatim}

\item \textit{Description}

Product types are the types of tuples. They are comprised of the expressions of
the types of the fields seperated by the string " \verb|x| " (space '\verb|x|'
space) because '\verb|x|' is very similar the symbol used in the cartesian
product. If any of the fields is of a product or a function type then the
corresponding type expression must be inside parentheses. A product type where
all the fields are of the same type can be abbreviated with a power type
expression which is comprised of the type, the power symbol '\verb|^|' and the
number of times the type is repeated.

\item \textit{Grammar}
\begin{grammar}
<prod-type> ::= <field-type> ( `\ x\ ' <field-type> )+

<field-type> ::= <power-base-type> | <power-type>

<power-base-type> ::=
<param-t-var> | <type-app-id-or-ahtv> |
`(' [ `\ ' ] ( <prod-type> | <func-type> ) [ `\ ' ] `)'

<power-type> ::= <power-base-type> `^' <int-greater-than-one>
\end{grammar}
\end{itemize}

\paragraph{Function Types}

\begin{itemize}
\item \textit{Examples}
\begin{verbatim}
String => String
Real => Int
T1 => T1
Int^2 => Int
Real^3 => Real
(T1 => T2) x (T2 => T3) => (T1 => T3)
(Int => Int) => (Int => Int)
\end{verbatim}

\item \textit{Description}

A function type expression is comprised of the input type expression and the
output type expression seperated by the function arrow ("\verb|=>|"). The
input and output type expressions are type expressions which are put in
parentheses only if they are function type expressions.

\item \textit{Grammar}
\begin{grammar}
<func-type> ::= <in-or-out-type> `\ =>\ ' <in-or-out-type>

<in-or-out-type> ::=
<param-t-var> | <type-app-id-or-ahtv> | <power-type> | <prod-type> |
`(' [ `\ ' ] <func-type>  [ `\ ' ] `)'
\end{grammar}
\end{itemize}

\newpage

\paragraph{Conditional Types}

\begin{itemize}
\item \textit{Examples}
\begin{verbatim}
(@A)And(@B)Can_Be_Equal --> @A x @B => Bool
(@A)And(@B)Add_To(@C) --> @A x @B => @C
(@A)Is(@B)s_First --> @B => @A
(@T)Has_Str_Rep --> @T => String
(@E)Has_Use --> @E(T1) x (T1 => @E(T2)) => @E(T2)
\end{verbatim}

\item \textit{Description}

Conditional types are the types of values that are polymorphic not because of
their structure but because they have been defined (seperately) for many
different combinations of types (i.e. they are ad hoc polymorphic). They are
comprised of a condition and a "simple" type (i.e. a type without a condition)
which are seperated by the condition arrow (" \verb|-->| "). The condition is
a type proposition which refers to type variables inside the "simple" type and
it must hold whenever the polymorphic value of that type is used. For example:
\begin{verbatim}
(_)first: (@A)Is(@B)s_First --> @B => @A
\end{verbatim}
can be used as follows:
\begin{verbatim}
pair, triple, list
  : Int x String, Real x Char x Int, ListOf(String)s
  = (42, "The answer to everything"), (3.14, 'a', 1), ["Hi!", "Hello", Heeey"]

>> (pair)first
  : Int
  ==> 42
>> (triple)first
  : Real
  ==> 3.14
>> (list)first
  : String
  ==> "Hi!"
\end{verbatim}
and that is because the following propositions hold:
\begin{verbatim}
(Int)Is(Int x String)s_First
(Real)Is(Real x Char x Int)s_First
(String)Is(ListOf(String)s)s_First
\end{verbatim}
which it turn means that the function "\verb|first|" has been defined for these
combinations of types. For more on how conditions, propositions and ad hoc
polymorphism works, see the "Type Logic" section (\ref{subsec:typelogic}).

\item \textit{Grammar}
\begin{grammar}
<condition> ::= <prop-name> `\ -->\ '
\end{grammar}

\end{itemize}

\newpage

\subsubsection{Type Definitions}
\label{subsubsec:tdefs}

Type definitions are divided into \verb|tuple_type| definitions and
\verb|or_type| definitions which are described in the following paragraphs.
\\\\
The grammar of a type definition is:
\begin{grammar}
<type-def> ::= <tuple-type-def> | <or-type-def>
\end{grammar}

\paragraph{Tuple Types}

\begin{itemize}
\item \textit{Definition Examples}

\begin{verbatim}
tuple_type Name
value (first_name, last_name) : String^2

tuple_type Date
value (day, month, year) : Int^3

tuple_type MathematicianInfo
value (name, nationality, date_of_birth) : Name x String x Date

tuple_type TreeOf(T1)s
value (root, subtrees) : T1 x ListOf(TreeOf(T1)s)s

tuple_type Indexed(T1)
value (index, val) : Int x T1
\end{verbatim}

\item \textit{Usage Examples}

\begin{verbatim}
euler_info: MathematicianInfo
  = (("Leonhard", "Euler"), "Swiss", (15, 4, 1707))

name(_)to_string: Name => String
  = n => "\nFirst Name: " + n.first_name + "\nLast Name: " + n.last_name

print_name_and_nat(_): MathematicianInfo => IO
  = ci => print(name(ci.name)to_string + "\nNationality: " + ci.nationality)

sum_nodes(_): TreeOf(Int)s => Int
  = tree => tree.root + apply(sum_nodes)to_all_in(tree.subtrees) -> sum_list
\end{verbatim}

\item \textit{Description}

A tuple type is equivalent to a product type with a new name  and names for the
fields for convinience. A tuple type generates postfix functions for all of
the fields by using a '.' before the name of the field. For example the
"\verb|MathematicianInfo|" type above generates the following functions:
\begin{verbatim}
_.name : MathematicianInfo => Name
_.nationality : MathematicianInfo => String
_.date_of_birth : MathematicianInfo => Date
\end{verbatim}

\newpage

\item \textit{Grammar}

\begin{grammar}
<tuple-type-def> ::= ""\\
`tuple_type\ ' <type-name> <nl> \\
`value' ( `\ ' | <nl> `\ \ ' )
<id-tuple> [ `\ ' ] `:' [ `\ ' ] ( <prod-type> | <power-type> )

<type-name> ::= ""\\""
[ <param-vars-in-paren> ] <type-id> ( <param-vars-in-paren> [A-Za-z]+ )*
[ <param-vars-in-paren> ]

<param-vars-in-paren> ::=
`(' [ `\ ' ] <param-t-var> ( <comma> <param-t-var> )* [ `\ ' ] `)'

<id-tuple> ::= `(' [ `\ ' ] <simple-id> ( <comma> <simple-id> )+ [ `\ ' ] `)'
\end{grammar}

\end{itemize}

\paragraph{Or Types}

\begin{itemize}
\item \textit{Definition Examples}

\begin{verbatim}
or_type Bool
values true | false

or_type Possibly(T1)
values the_value:T1 | no_value

// needed tuple_type for ListOf(T1)s
tuple_type NonEmptyListOf(T1)s
value (head, tail) : T1 x ListOf(T1)s

or_type ListOf(T1)s
values non_empty_l:NonEmptyListOf(T1)s | empty_l

or_type Result(T1)OrError(T2)
values result:T1 | error:T2
\end{verbatim}

\item \textit{Usage Examples}

\begin{verbatim}
(_)is_empty: ListOf(T1)s => Bool
  = cases
    empty_l => true
    non_empty_l:* => false

(_)head: ListOf(T1)s => Possibly(T1)
  = cases
    empty_l => no_value
    non_empty_l:list => the_value:list.head

sum_list(_): ListOf(Int)s => Int
  = cases
    empty_l => 0
    non_empty_l:l => l.head + sum_list(l.tail)

print_err(_)or_res(_): (@A)Has_Str_Rep --> Result(@A)OrError(String) => IO
  = cases
    result:r => print("All good! The result is: " + (r)to_string)
    error:e => print("Error occured: " + e)
\end{verbatim}

\item \textit{Description}

The values of an \verb|or_type| are split into cases. Some cases have other
values inside.  The cases which have other values inside are followed by a
colon and the type of the internal value. Similar syntax can be used for
matching that particular case in a function using the "cases" syntax.  An
\verb|or_type| definition automatically creates prefix functions for each
case with an internal value (which are simply conversions from the type of the
internal value to the \verb|or_type|).  For example, for the case
"\verb|non_empty_l|" of a list, the function "\verb|non_empty_l:_|" is
automatically created from the definition for which we can say:
\begin{verbatim}
non_empty_l:_ : NonEmptyListOf(T1)s => ListOf(T1)s
\end{verbatim}
For example:
\begin{verbatim}
non_e_l : NonEmptyListOf(Int)s
  = (1, [2, 3, 4])
>> non_empty_l:non_e_l
  : ListOf(Int)s
  ==> [1, 2, 3, 4]
\end{verbatim}
Similarly:
\begin{verbatim}
the_value:_ : T1 => Possibly(T1)
\end{verbatim}
These functions can be used like any other function as arguments to other
functions.  For example:
\begin{verbatim}
non_empty_ls(_)to_ls : ListOf(NonEmptyListOf(T1)s)s => ListOf(ListOf(T1)s)s
  = apply(non_empty_l:_)to_all_in(_)
\end{verbatim}

\item \textit{Grammar}

\begin{grammar}
<or-type-def> ::= ""\\
`or_type\ ' <type-name> <nl> \\
`values' ( `\ ' | <nl> `\ \ ' ) \\
<simple-id> [ `:' <simple-type> ]
( [ `\ ' ] `|' [ `\ ' ] <simple-id> [ `:' <simple-type> ] )*
\end{grammar}

\end{itemize}

\subsubsection{Type Nicknames}

\begin{itemize}

\item \textit{Examples}

\begin{verbatim}
type_nickname Ints = ListOf(Int)s
type_nickname IntStringPairs = ListOf(Int x String)s
type_nickname IO = (EmptyVal)WithIO
type_nickname Res(T1)OrErr = Result(T1)OrError(String)
\end{verbatim}

\item \textit{Description} \\\\
Type nicknames are used to abbreviate or give a more descriptive name to a
type.  They start with the keyword "\verb|type_nickname|", followed by the
nickname, then an equal sign and they end with the type to be nicknamed.
Parametric type variables can be used in the nickname.

\item \textit{Grammar}

\begin{grammar}
<t-nickname> ::=
`type_nickname\ ' <type-name> <equals> <simple-type>
\end{grammar}

\end{itemize}

\subsection{Type Logic}
\label{subsec:typelogic}

Type logic is the mechanism for ad hoc polymorphism in lcases. The central
notion of \textbf{type logic} is the \textbf{type proposition}. A type
proposition is a proposition that has types as parameters and is true or false
for particular type arguments.
\\\\
Type propositions can either be defined or proven (for certain type arguments).
Therefore, the following constructs exist and accomplish the aforementioned
respectively: \textbf{type proposition definitions} and \textbf{type theorems}.
These constructs are described in detail in the following sections. From this
point onwards the "type" part will be omitted, i.e. propositions are always
type propositions and theorems are always type theorems.

\subsubsection{Proposition Definitions}

Proposition definitions are split into definitions of \textbf{atomic
propositions} and definitions of \textbf{renaming propositions} which are
described in the following paragraphs.

\paragraph{Atomic Propositions}

\begin{itemize}
\item \textit{Examples}
\begin{verbatim}
type_proposition (@A)Is(@B)s_First
needed (_)first: @B => @A

type_proposition (@T)Has_Str_Rep
needed (_)to_string: @T => String

type_proposition (@T)Has_A_Wrapper
needed wrap(_): T1 => @T(T1)

type_proposition (@T)Has_Internal_App
needed apply(_)inside(_) : (T1 => T2) x @T(T1) => @T(T2)
\end{verbatim}
The examples above define the following (ad hoc) polymorphic functions which
have the respective (conditional) types:
\begin{verbatim}
(_)first: (@A)Is(@B)s_First --> @B => @A

(_)to_string: (@T)Has_Str_Rep --> @T => String

wrap(_): (@T)Has_A_Wrapper --> T1 => @T(T1)

apply(_)inside(_): (@T)Has_Internal_App --> (T1 => T2) x @T(T1) => @T(T2)
\end{verbatim}

\newpage

\item \textit{Description}

An atomic proposition definition defines simultaneously the \textbf{atomic
proposition} itself and a \textbf{polymorphic value} (usually, but not
necessarily, a function), by definining the form of the type of the value given
the type parameters of the proposition. The proposition is true or false when
the type parameters are substituted by specific type arguments depending on
whether the implementation of the value has been defined for these type
arguments. The aforementioned truthvalue determines whether the value is used
correctly inside the program and therefore whether the program will typecheck.
In order to add more types for which the function works, i.e.  define the
function for these types, i.e. make the proposition true for these types, one
must prove a theorem. The specifics of theorems are described in the next
section. For now, we'll show the example for everything mentioned in this
paragraph for the proposition "\verb|(@A)Is(@B)s_First|":

\begin{itemize}
\item Proposition Definition:

\begin{verbatim}
type_proposition (@A)Is(@B)s_First
needed (_)first: @B => @A
\end{verbatim}

\item Function defined and its type:

\begin{verbatim}
(_)first: (@A)Is(@B)s_First --> @B => @A
\end{verbatim}

\item Theorems for specific types:

\begin{verbatim}
type_theorem (T1)Is(T1 x T2)s_First
proof (_)first = _.1st

type_theorem (T1)Is(ListOf(T1)s)s_First
proof
  (_)first =
    cases
      empty_l => show_err("Tried to take the first element of an empty list")
      non_empty_l:l => l.head
\end{verbatim}

\item Usage of the function
\begin{verbatim}
pair, list
  : Int x String, ListOf(String)s
  = (42, "The answer to everything"), ["Hi!", "Hello", Heeey"]

>> (pair)first
  : Int
  ==> 42
>> (list)first
  : String
  ==> "Hi!"
\end{verbatim}
\end{itemize}

An atomic proposition definition begins with the keyword
"\verb|type_proposition|" followed by the name of the proposition (including
the type parameters) in the first line. The second line begins with the keyword
"\verb|needed|" which is followed by the identifier and the
type expression of the value seperated by the "has type" symbol ('\verb|:|').

\end{itemize}

\newpage

\paragraph{Renaming Propositions}

\begin{itemize}
\item \textit{Examples}
\begin{verbatim}
type_proposition (@T)Has_Equality
equivalent (@T)And(@T)Can_Be_Equal

type_proposition (@A)And(@B)Are_Comparable
equivalent
  (@A)Can_Be_Less_Than(@B), (@A)And(@B)Can_Be_Equal, (@A)Can_Be_Greater_Than(@B)

type_proposition (@T)Has_Comparison
equivalent (@T)And(@T)Are_Comparable
\end{verbatim}

\item \textit{Description}

A renaming proposition definition is used to abbreviate one or the conjunction
of many propositions (i.e. AND of all of them) into one new proposition.
\\\\
A renaming proposition definition begins with the keyword
"\verb|type_proposition|" followed by the name of the proposition (including
the type parameters) in the first line. The second line begins with the keyword
"\verb|equivalent|" followed by either one proposition or (if it is a
conjunction) many propositions seperated by commas (where the commas
essentially mean "and").

\end{itemize}

\paragraph{Grammar for Proposition Definitions}

\begin{grammar}
<type-prop-def> ::= <atom-prop-def> | <renaming-prop-def>
\\

<atom-prop-def> ::=
<prop-name-line> <nl> `needed' ( `\ ' | <nl> `\ \ ' )
<identifier> [ `\ ' ] `:' [ `\ ' ] <simple-type>

<renaming-prop-def> ::= ""\\
<prop-name-line> <nl>
`equivalent' ( `\ ' | <nl> `\ \ ' ) <prop-name> ( <comma> <prop-name> )*

<prop-name-line> ::= `type_proposition\ ' <prop-name>

<prop-name> ::=  ""\\""
[A-Z] ( <name-part> <types-in-paren> )+ [ <name-part> ]
\alt " "( <types-in-paren> <name-part> )+ [ <types-in-paren> ]

<name-part> ::= ( [A-Za-z] | `_'[A-Z] )+
\end{grammar}

\newpage

\subsubsection{Theorems}
\label{subsubsec:ttheo}

Theorems are split into theorems of \textbf{atomic propositions} and
theorems of \textbf{implication propositions} which are described in the
following paragraphs.

\paragraph{Atomic Propositions}

\begin{itemize}
\item \textit{Examples}

\begin{verbatim}
type_theorem (Possibly(_))Has_A_Wrapper
proof wrap(_) = the_value:_

type_theorem (ListOf(_)s)Has_A_Wrapper
proof wrap(_) = [_]

type_theorem (Possibly(_))Has_Internal_App
proof
  apply(_)inside(_) =
    (f(_), cases)
      no_value => no_value
      the_value:x => the_value:f(x)

type_theorem (ListOf(_)s)Has_Internal_App
proof apply(_)inside(_) = apply(_)to_all_in(_)
\end{verbatim}

\item \textit{Usage}

\begin{verbatim}
a, b : all Possibly(Int)
  = wrapper(1), no_value

l1, l2, l3 : all ListOf(Int)s
  = wrapper(1), empty_l, [1, 2, 3]

>> a
  : Possibly(Int)
  ==> the_value:1
>> b
  : Possibly(Int)
  ==> no_value
>> l1
  : ListOf(Int)s
  ==> [1]
>> l2
  : ListOf(Int)s
  ==> []
\end{verbatim}
\newpage
\begin{verbatim}
>> apply(_ + 1)inside(a)
  : Possibly(Int)
  ==> the_value:2
>> apply(_ + 1)inside(b)
  : Possibly(Int)
  ==> no_value
>> apply(_ + 1)inside(l1)
  : ListOf(Int)s
  ==> [2]
>> apply(_ + 1)inside(l2)
  : ListOf(Int)s
  ==> []
>> apply(_ + 1)inside(l3)
  : ListOf(Int)s
  ==> [2, 3, 4]
\end{verbatim}

\item \textit{Description}

A theorem of an atomic proposition proves the proposition for
specific type arguments, by implementing the value associated to the
proposition for these type arguments. Therefore, the value associated with the
proposition can be used with all the combinations of type arguments for which
the proposition is true, i.e. the combinations of type arguments for which
the value has been implemented.
\\\\
A proof of a theorem of an atomic proposition is correct when the
implementation of the value associated with the proposition follows the form of
the type given to the value by the definition of the proposition, i.e. the only
difference between the type of the value in the theorem and the type of the
value in the definition is that the type parameters of the proposition are
substituted by the type arguments of the theorem.
\\\\
A theorem of an atomic proposition begins with the keyword
"\verb|type_theorem|" followed by the name of the proposition with the type
parameters substituted by the specific types for which the proposition will be
proven. The second line is the keyword "\verb|proof|". The third line is
indented once and it is the line in which the proof begins. The proof begins
with the identifier of the value associated with the proposition and is
followed by an equal sign and the value expression which implements the value.

\end{itemize}

\newpage

\paragraph{Implication Propositions}

\begin{itemize}
\item \textit{Examples}

\begin{verbatim}
type_theorem (@A)And(@B)Can_Be_Equal --> (@A)And(@B)Can_Be_Unequal
proof a \= b = not(a == b)

type_theorem (@A)Can_Be_Greater_Than(@B) --> (@A)Can_Be_Le_Or_Eq_To(@B)
proof a <= b = not(a > b)

type_propositon (@A)And(@B)Have_Eq_And_Gr
equivalent (@A)And(@B)Can_Be_Equal, (@A)Can_Be_Greater_Than(@B)

type_theorem (@A)And(@B)Have_Eq_And_Gr --> (@A)Can_Be_Gr_Or_Eq_To(@B)
proof a >= b = a == b | a > b
\end{verbatim}

\item \textit{Description}

A theorem of an implication proposition is very similar to a theorem of an
atomic proposition in the sense that it also implements a value in the proof.
The difference is that the implementation uses another ad hoc polymorphic value
(or many). Therefore, the implementation does not prove the proposition
associated to the value it implements, because it assumes that the polymorphic
value(s) used in the implementation is(are) already defined. In other words it
proves the following: "if this(these) ad hoc polymorphic value(s) is(are)
defined then we can also define this other one". This can be translated into
the following implication proposition: "if the proposition associated to the
value(s) we are using is true then the proposition associated to the value we
are defining is true", which can be condensed to the notation with the
condition arrow (" \verb|-->| ") used in the examples.
\\\\
The proof of an implication proposition allows the compiler to
automatically create the definition for an ad hoc polymorphic value for a
particular combination of types given the definitions of the ad hoc polymorphic
values used in the implementation for this same combination of types. This
mechanism essentially gives definitions for free, that is in the sense that
when you define a set of ad hoc polymorphic values for a particular set of
types you get for free all the ad hoc polymorphic values that can be defined
using a subset of the defined ones.
\\\\
A theorem of an implication proposition is grammatically the same as
a theorem of an atomic proposition with the only difference being that an
implication propostition is comprised by two atomic propositions seperated by
the condition arrow (" \verb|-->| ") arrow.

\end{itemize}

\newpage

\paragraph{Grammar for Theorems}

\begin{grammar}
<type-theo> ::=
`type_theorem\ ' <prop-name-with-subs> [ `\ -->\ ' <prop-name-with-subs> ] <nl>
`proof' <proof>

<prop-name-with-subs> ::=  ""\\""
[A-Z] ( <name-part> <subs-in-paren> )+ [ <name-part> ]
\alt " "( <subs-in-paren> <name-part> )+ [ <subs-in-paren> ]

<subs-in-paren> ::=
`(' [ `\ ' ] <t-var-sub> ( <comma> <t-var-sub> )* [ `\ ' ] `)'
\\

<t-var-sub> ::=
<param-t-var> | <type-app-id-or-ahtv-sub> | <power-type-sub> | <prod-type-sub> |
<func-type-sub>
\\

<type-app-id-or-ahtv-sub> ::=
[ <subs-or-unders-in-paren> ] <taioas-middle> [ <subs-or-unders-in-paren> ]

<taioas-middle> ::=
<type-id> ( <subs-or-unders-in-paren> [A-Za-z]+ )* | <ad-hoc-t-var>

<subs-or-unders-in-paren> ::=
`(' [ `\ ' ] <sub-or-under> ( <comma> <sub-or-under> )* [ `\ ' ] `)'

<sub-or-under> ::= <t-var-sub> | `_'
\\

<power-type-sub> ::= <power-base-type-sub> `^' <int-greater-than-one>

<power-base-type-sub> ::= ""\\
`_' | <param-t-var> | <type-app-id-or-ahtv-sub> |
`(' [ `\ ' ] ( <prod-type-sub> | <func-type-sub> ) [ `\ ' ] `)'
\\

<prod-type-sub> ::= <field-type-sub> ( `\ x\ ' <field-type-sub> )+

<field-type-sub> ::= <power-base-type-sub> | <power-type-sub>
\\

<func-type-sub> ::= <in-or-out-type-sub> `\ =>\ ' <in-or-out-type-sub>

<in-or-out-type-sub> ::= ""\\
`_' | <param-t-var> | <type-app-id-or-ahtv-sub> | <power-type-sub> |
<prod-type-sub> | \\`(' [ `\ ' ] <func-type-sub> [ `\ ' ] `)'
\\

<proof> ::=
`\ ' <id-or-op-eq> `\ ' <line-expr> |
<nl> `\ \ ' <id-or-op-eq>  <tt-value-expr>

<id-or-op-eq> ::= <identifier> [ <op> <identifier> ] `\ ='

<tt-value-expr> ::=
`\ ' <line-expr> | <nl> <indent> <value-expr> [ <where-expr> ]
\end{grammar}

\newpage

\section{Language Description: Predefined}

\subsection{Values}

\begin{itemize}

\item Constants: \verb|undefined, pi|

\item Functions
  \begin{itemize}

  \item Miscellaneous: \verb|not(_), id(_), throw_err(_)|

  \item Numerical:
    \begin{itemize}
    \item Miscellaneous:
      \verb|sqrt_of(_), abs_val_of(_), max_of(_)and(_), min_of(_)and(_)|
    \item Trigonometric:
      \verb|sin(_), cos(_), tan(_), asin(_), acos(_), atan(_)|
    \item Division related: \\
      \verb|    (_)div(_), (_)mod(_), gcd_of(_)and(_), lcm_of(_)and(_),|
      \verb|(_)is_even, (_)is_odd|
    \item Rounding:
      \verb|truncate(_), round(_), floor(_), ceiling(_)|
    \item e and log:
      \verb|exp(_), ln(_), log_of(_)base(_)|
    \end{itemize}

  \item List:\\
    \verb|(_)length, (_)is_in(_), apply(_)to_all_in(_), filter(_)with(_), |
    \\
    \verb|take(_)from(_), ignore(_)from(_), zip(_)and(_), unzip(_), |
    \verb|zip(_)and(_)with(_)|

  \item IO:
    \begin{itemize}
    \item Input: \verb|get_char, get_line, get_input, read_file(_)|
    \item Output:
      \verb|print(_), print_string(_), print_line(_), write(_)in_file(_)|
    \end{itemize}

  \item Ad Hoc Polymorphic:\\
  \verb|(_)first, (_)second, (_)third, (_)fourth, (_)fifth, wrap(_), |
  \\
  \verb|(_)to_string, from_string(_), apply(_)inside(_), |
  \\
  \verb|apply_wrapped(_)inside(_), (_)to_list, from_list(_)|

  \end{itemize}
\end{itemize}

\subsection{Types}

\begin{itemize}
  \item
  Basic: \verb|Int, Real, Char, String|
  \item
  Or Types: \verb|EmptyVal, Bool, Possibly(_), ListOf(_)s, Result(_)OrError(_)|
  \item
  Tuple Types: \verb|NonEmptyListOf(_)s|
  \item
  Type Nicknames: \verb|IO, Z, R, C|
\end{itemize}

\newpage

\subsection{Type Propositions}

\begin{itemize}

\item Operator Propositions:

  \begin{itemize}
  \item \verb|(@A)To_The(@B)Is(@C)|
  \item \verb|(@A)And(@B)Multiply_To(@C)|
  \item \verb|(@A)Divided_By(@B)Is(@C)|
  \item \verb|(@A)And(@B)Add_To(@C)|
  \item \verb|(@A)Minus(@B)Is(@C)|
  \item \verb|(@A)And(@B)Can_Be_Equal|
  \item \verb|(@A)And(@B)Can_Be_Unequal|
  \item \verb|(@A)Can_Be_Gr_Or_Eq_To(@B)|
  \item \verb|(@A)Can_Be_Le_Or_Eq_To(@B)|
  \item \verb|(@A)Can_Be_Greater_Than(@B)|
  \item \verb|(@A)Can_Be_Less_Than(@B)|
  \item \verb|(@A)Has_And|
  \item \verb|(@A)Has_Or|
  \item \verb|(@T)Has_Use|
  \item \verb|(@T)Has_Then|
  \end{itemize}

\item Function Propositions:

  \begin{itemize}
  \item \verb|(@A)Is(@B)s_First|
  \item \verb|(@A)Is(@B)s_Second|
  \item \verb|(@A)Is(@B)s_Third|
  \item \verb|(@A)Is(@B)s_Fourth|
  \item \verb|(@A)Is(@B)s_Fifth|
  \item \verb|(@T)Has_A_Wrapper|
  \item \verb|(@T)Has_Str_Rep|
  \item \verb|(@T)Can_Be_Parsed|
  \item \verb|(@T)Has_Internal_App|
  \item \verb|(@T)Has_Wrapped_Internal_App|
  \end{itemize}

\item Renaming Propositions:

  \begin{itemize}
  \item \verb|(@T)Has_Exponentiation|
  \item \verb|(@T)Has_Multiplication|
  \item \verb|(@T)Has_Division|
  \item \verb|(@T)Has_Addition|
  \item \verb|(@T)Has_Subtraction|
  \item \verb|(@T)Has_Equality|
  \item \verb|(@T)Has_Non_Equality|
  \item \verb|(@T)Has_Greater_Or_Equal|
  \item \verb|(@T)Has_Less_Or_Equal|
  \item \verb|(@T)Has_Greater|
  \item \verb|(@T)Has_Less|
  \end{itemize}

\item Theorems:

  \begin{itemize}
  \item \verb|(T1)Is(T1 x T2)s_First|
  \item \verb|(T1)Is(T1 x T2 x T3)s_First|
  \item \verb|(T1)Is(T1 x T2 x T3 x T4)s_First|
  \item \verb|(T1)Is(T1 x T2 x T3 x T4 x T5)s_First|
  \item \verb|(T1)Is(ListOf(T1)s)s_First|
  \item \verb|(T2)Is(T1 x T2)s_Second|
  \item \verb|(T2)Is(T1 x T2 x T3)s_Second|
  \item \verb|(T2)Is(T1 x T2 x T3 x T4)s_Second|
  \item \verb|(T2)Is(T1 x T2 x T3 x T4 x T5)s_Second|
  \item \verb|(T1)Is(ListOf(T1)s)s_Second|
  \item \verb|(T3)Is(T1 x T2 x T3)s_Third|
  \item \verb|(T3)Is(T1 x T2 x T3 x T4)s_Third|
  \item \verb|(T3)Is(T1 x T2 x T3 x T4 x T5)s_Third|
  \item \verb|(T1)Is(ListOf(T1)s)s_Third|
  \item \verb|(T4)Is(T1 x T2 x T3 x T4)s_Fourth|
  \item \verb|(T4)Is(T1 x T2 x T3 x T4 x T5)s_Fourth|
  \item \verb|(T1)Is(ListOf(T1)s)s_Fourth|
  \item \verb|(T5)Is(T1 x T2 x T3 x T4 x T5)s_Fifth|
  \item \verb|(T1)Is(ListOf(T1)s)s_Fifth|
  \item \verb|TODO wrapper|
  \item \verb|(Int)Has_Str_Rep|
  \item \verb|(Char)Has_Str_Rep|
  \item \verb|(Real)Has_Str_Rep|
  \item \verb|(@A)Has_Str_Rep --> (ListOf(@A)s)Has_Str_Rep|
  \item \verb|TODO apply()inside|
  \item \verb|TODO apply_wrapped()inside|
  \item \verb|TODO ;>|
  \item \verb|TODO ;|
  \end{itemize}

\end{itemize}

\newpage

\section{Parser Implementation}

\subsection{Full grammar and indentation system}

\subsubsection{Full grammar}

\begin{grammar}

<literal> ::= <int-lit> | <real-lit> | <char-lit> | <string-lit>
\\

<identifier> ::=
[ <unders-in-paren> ] <id-start> <id-cont>* [ [0-9] ] [ <unders-in-paren> ]

<simple-id> ::= <id-start> [ [0-9] ]

<id-start> ::= [a-z] [a-z_]*

<id-cont> ::= <unders-in-paren> [a-z_]+

<unders-in-paren> ::= `(_' ( <comma> `_' )* `)'

<comma> ::= `,' [ `\ ' ]
\\

<paren-expr> ::= `(' [ `\ ' ] <line-op-expr> | <line-func-expr> [ `\ ' ] `)'
\\

<tuple> ::=
`(' [ `\ ' ] <line-expr-or-under> <comma> <line-expr-or-unders> [ `\ ' ] `)'

<line-expr-or-unders> ::=
<line-expr-or-under> ( <comma> <line-expr-or-under> )*

<line-expr-or-under> ::= <line-expr> | `_'

<line-expr> ::= <basic-or-app-expr> | <line-op-expr> | <line-func-expr>

<basic-or-app-expr> ::= <basic-expr> | <pre-func-app> | <post-func-app>

<basic-expr> ::=
<literal> | <paren-func-app-or-id> | <special-id> | <tuple> | <list>
\\

<big-tuple> ::= ""\\
`(' [ `\ ' ] <line-expr-or-under> [ <nl> <indent> ]
<comma> <line-expr-or-unders> \\
( <nl> <indent> <comma> <line-expr-or-unders> )*
<nl> <indent> `)'
\\

<list> ::= `[' [ `\ ' ] [ <line-expr-or-unders> ] [ `\ ' ] `]'

<big-list> ::=
`[' [ `\ ' ] <line-expr-or-unders>
( <nl> <indent> <comma> <line-expr-or-unders> )* <nl> <indent> `]'
\\

<paren-func-app-or-id> ::=
[ <arguments> ] <id-start> ( <arguments> [a-z_]+ )* [ [0-9] ]
[ <arguments> ]

<arguments> ::= `(' [ `\ ' ] <line-expr-or-unders> [ `\ ' ] `)'
\\

\newpage
<pre-func> ::= <simple-id> `:'

<pre-func-app> ::= <pre-func> <operand>
\\

<post-func> ::= `.' ( <simple-id> | <special-id> )

<special-id> ::= `1st' | `2nd' | `3rd' | `4th' | `5th'

<post-func-app> ::=
( <basic-expr> | <paren-expr> | `_' )
( <dot-change> | <post-func>+ [ <dot-change> ] )
\\

<dot-change> ::=
`.change{' [ `\ ' ] <field-change> ( <comma> <field-change> )* [ `\ ' ] `}'

<field-change> ::= ( <simple-id> | <special-id> ) <equals> <line-expr-or-under>

<equals> ::= [ `\ ' ] `=' [ `\ ' ]
\\

<op-expr> ::= <line-op-expr> | <big-op-expr>
\\

<op-expr-start> ::= ( <operand> <op> )+

<line-op-expr> ::= <op-expr-start> ( <operand> | <line-func-expr> )
\\

<big-op-expr> ::=
<big-op-expr-op-split> | <big-op-expr-func-split>
\\

<big-op-expr-op-split> ::=
<op-split-line>+ [ <op-expr-start> ] ( <operand> | <func-expr> )

<op-split-line> ::=
( <op-expr-start> ( <nl> | <oper-fco> ) | <oper-fco> ) <indent>

<oper-fco> ::= <operand> `\ ' <func-comp-op> `\\n'
\\

<big-op-expr-func-split> ::=
<op-expr-start> ( <big-func-expr> | <cases-func-expr> )
\\

<operand> ::= <basic-or-app-expr> | <paren-expr> | `_'
\\

<op> ::= `\ ' <func-comp-op> `\ ' | [ `\ ' ] <optional-spaces-op> [ `\ ' ]

<func-comp-op> ::= `o>' | `<o'

<optional-spaces-op> ::=
`->' | `<-' | `^' | `*' | `/' | `+' | `-' | `==' | `!=' | `>' | `<' | `>=' |
`<=' | `\&' | `|' | `;>' | `;'
\\

\newpage
<func-expr> ::= <line-func-expr> | <big-func-expr> | <cases-func-expr>
\\

<line-func-expr> ::= <parameters> [ `\ ' ] `=>' <line-func-body>

<big-func-expr> ::= <parameters> [ `\ ' ] `=>' <big-func-body>

<parameters> ::=
<identifier> | `*' |
`(' [ `\ ' ] <parameters> ( <comma> <parameters> )+ [ `\ ' ] `)'

<line-func-body> ::=
[ `\ ' ]
(
<basic-or-app-expr> | <line-op-expr> |
`(' [ `\ ' ] <line-func-expr> [ `\ ' ] `)'
)

<big-func-body> ::=
<nl> <indent>
(
<basic-or-app-expr> | <op-expr> | `(' [ `\ ' ] <line-func-expr> [ `\ ' ] `)'
)
\\

<cases-func-expr> ::= <cases-params> <case>+ [ <end-case> ]

<cases-params> ::=
<identifier> | `cases' | `*' |
`(' [ `\ ' ] <cases-params> ( <comma> <cases-params> )+ [ `\ ' ] `)'
\\

<case> ::=  <nl> <indent> <outer-matching> [ `\ ' ] `=>' <case-body>

<end-case> ::=
<nl> <indent> ( `...' | <identifier> ) [ `\ ' ] `=>' <case-body>
\\

<outer-matching> ::= <simple-id> | <matching>

<matching> ::=
<literal> | <pre-func> <inner-matching> | <tuple-matching> | <list-matching>

<inner-matching> ::= `*' | <identifier> | <matching>

<tuple-matching> ::=
`(' [ `\ ' ] <inner-matching> ( <comma> <inner-matching> )+ [ `\ ' ] `)'

<list-matching> ::=
`[' [ `\ ' ]
[ <inner-matching> ( <comma> <inner-matching> )* [ <rest-list-matching> ] ]
[ `\ ' ] `]'

<rest-list-matching> ::= <comma> [ <simple-id> <equals> ] `...'
\\

<case-body> ::= <line-func-body> | <big-func-body> [ <where-expr> ]
\\

<value-def> ::= ""\\
<indent> <identifier>
( [ `\ ' ] `:' [ `\ ' ] | <nl> <indent> `:\ ' ) <type>  \\
<nl> <indent> `=\ ' <value-expr> [ <where-expr> ]

<value-expr> ::=
<basic-or-app-expr> | <op-expr> | <func-expr> | <big-tuple> | <big-list>
\\

<grouped-value-defs> ::= ""\\
<indent> <identifier> ( <comma> <identifier> )+ \\
( [ `\ ' ] `:' [ `\ ' ] | <nl> <indent> `:\ ' )
( <type> ( <comma> <type> )+ | `all\ ' <type> ) \\
<nl> <indent> `=\ ' <line-exprs> ( <nl> <indent> <comma> <line-exprs> )*

<line-exprs> ::= <line-expr> ( <comma> <line-expr> )*
\\

<where-expr> ::=
<nl> <indent> `where'
<nl> <value-def-or-defs> ( <nl> <nl> <value-def-or-defs> )*

<value-def-or-defs> ::= <value-def> | <grouped-value-defs>
\\

<type> ::= [ <condition> ]  <simple-type>

<simple-type> ::=
<param-t-var> | <type-app-id-or-ahtv> | <power-type> | <prod-type> |
<func-type>
\\

<type-id> ::= [A-Z] [A-Za-z]*
\\

<param-t-var> ::= `T' [0-9]

<ad-hoc-t-var> ::= `@' [A-Z]
\\

<type-app-id-or-ahtv> ::=
[ <types-in-paren> ] <taioa-middle> [ <types-in-paren> ]

<taioa-middle> ::=
<type-id> ( <types-in-paren> [A-Za-z]+ )* | <ad-hoc-t-var>

<types-in-paren> ::=
`(' [ `\ ' ] <simple-type> ( <comma> <simple-type> )* [ `\ ' ] `)'
\\

<prod-type> ::= <field-type> ( `\ x\ ' <field-type> )+

<field-type> ::= <power-base-type> | <power-type>

<power-base-type> ::=
<param-t-var> | <type-app-id-or-ahtv> |
`(' [ `\ ' ] ( <prod-type> | <func-type> ) [ `\ ' ] `)'

<power-type> ::= <power-base-type> `^' <int-greater-than-one>
\\

<func-type> ::= <in-or-out-type> `\ =>\ ' <in-or-out-type>

<in-or-out-type> ::=
<param-t-var> | <type-app-id-or-ahtv> | <power-type> | <prod-type> |
`(' [ `\ ' ] <func-type>  [ `\ ' ] `)'
\\

<condition> ::= <prop-name> `\ -->\ '
\\
\\

\newpage
<type-def> ::= <tuple-type-def> | <or-type-def>
\\

<tuple-type-def> ::= ""\\
`tuple_type\ ' <type-name> <nl> \\
`value' ( `\ ' | <nl> `\ \ ' )
<id-tuple> [ `\ ' ] `:' [ `\ ' ] ( <prod-type> | <power-type> )

<type-name> ::= ""\\""
[ <param-vars-in-paren> ] <type-id> ( <param-vars-in-paren> [A-Za-z]+ )*
[ <param-vars-in-paren> ]

<param-vars-in-paren> ::=
`(' [ `\ ' ] <param-t-var> ( <comma> <param-t-var> )* [ `\ ' ] `)'

<id-tuple> ::= `(' [ `\ ' ] <simple-id> ( <comma> <simple-id> )+ [ `\ ' ] `)'
\\

<or-type-def> ::= ""\\
`or_type\ ' <type-name> <nl> \\
`values' ( `\ ' | <nl> `\ \ ' ) \\
<simple-id> [ `:' <simple-type> ]
( [ `\ ' ] `|' [ `\ ' ] <simple-id> [ `:' <simple-type> ] )*
\\

<t-nickname> ::=
`type_nickname\ ' <type-name> <equals> <simple-type>
\\\\

<type-prop-def> ::= <atom-prop-def> | <renaming-prop-def>
\\

<atom-prop-def> ::=
<prop-name-line> <nl> `needed' ( `\ ' | <nl> `\ \ ' )
<identifier> [ `\ ' ] `:' [ `\ ' ] <simple-type>

<renaming-prop-def> ::= ""\\
<prop-name-line> <nl>
`equivalent' ( `\ ' | <nl> `\ \ ' ) <prop-name> ( <comma> <prop-name> )*

<prop-name-line> ::= `type_proposition\ ' <prop-name>

<prop-name> ::=  ""\\""
[A-Z] ( <name-part> <types-in-paren> )+ [ <name-part> ]
\alt " "( <types-in-paren> <name-part> )+ [ <types-in-paren> ]

<name-part> ::= ( [A-Za-z] | `_'[A-Z] )+
\\

\newpage
<type-theo> ::=
`type_theorem\ ' <prop-name-with-subs> [ `\ -->\ ' <prop-name-with-subs> ] <nl>
`proof' <proof>

<prop-name-with-subs> ::=  ""\\""
[A-Z] ( <name-part> <subs-in-paren> )+ [ <name-part> ]
\alt " "( <subs-in-paren> <name-part> )+ [ <subs-in-paren> ]

<subs-in-paren> ::=
`(' [ `\ ' ] <t-var-sub> ( <comma> <t-var-sub> )* [ `\ ' ] `)'
\\

<t-var-sub> ::=
<param-t-var> | <type-app-id-or-ahtv-sub> | <power-type-sub> | <prod-type-sub> |
<func-type-sub>
\\

<type-app-id-or-ahtv-sub> ::=
[ <subs-or-unders-in-paren> ] <taioas-middle> [ <subs-or-unders-in-paren> ]

<taioas-middle> ::=
<type-id> ( <subs-or-unders-in-paren> [A-Za-z]+ )* | <ad-hoc-t-var>

<subs-or-unders-in-paren> ::=
`(' [ `\ ' ] <sub-or-under> ( <comma> <sub-or-under> )* [ `\ ' ] `)'

<sub-or-under> ::= <t-var-sub> | `_'
\\

<power-type-sub> ::= <power-base-type-sub> `^' <int-greater-than-one>

<power-base-type-sub> ::= ""\\
`_' | <param-t-var> | <type-app-id-or-ahtv-sub> |
`(' [ `\ ' ] ( <prod-type-sub> | <func-type-sub> ) [ `\ ' ] `)'
\\

<prod-type-sub> ::= <field-type-sub> ( `\ x\ ' <field-type-sub> )+

<field-type-sub> ::= <power-base-type-sub> | <power-type-sub>
\\

<func-type-sub> ::= <in-or-out-type-sub> `\ =>\ ' <in-or-out-type-sub>

<in-or-out-type-sub> ::= ""\\
`_' | <param-t-var> | <type-app-id-or-ahtv-sub> | <power-type-sub> |
<prod-type-sub> | \\`(' [ `\ ' ] <func-type-sub> [ `\ ' ] `)'
\\

<proof> ::=
`\ ' <id-or-op-eq> `\ ' <line-expr> |
<nl> `\ \ ' <id-or-op-eq>  <tt-value-expr>

<id-or-op-eq> ::= <identifier> [ <op> <identifier> ] `\ ='

<tt-value-expr> ::=
`\ ' <line-expr> | <nl> <indent> <value-expr> [ <where-expr> ]
\\

<program> ::=
<nl>* <program-part> ( <nl> <nl> <program-part> )* <nl>*

<program-part> ::=
<value-def> | <grouped-value-defs> | <type-def> | <t-nickname> |
<type-prop-def> | <type-theo>

<nl> :: ( `\ ' | `\\t' )* `\\n'
\end{grammar}

\newpage

\subsubsection{Indentation system}
\label{subsubsec:indsys}

The \textit{$<$indent$>$} nonterminal in not a normal BNF nonterminal. It is a
context sensitive construct that enforces the indentation rules of lcases.
It depends on a integer value called the "indentation level" ($il$).
The \textit{$<$indent$>$} nonterminal corresponds to $2*il$ space characters.
The indentation level follows the rules below:

\paragraph{Indentation Rules}
\begin{enumerate}

\item
At the beginnng: $il = 0$

\item
In a single value definition:
  \begin{enumerate}

  \item
  At the end of the first line: $il \leftarrow il + 1$

  \item
  At the end of the "=" line: $il \leftarrow il + 1$

  \item
  At the end: $il \leftarrow il - 2$
  \end{enumerate}

\item
In a group of value definitions:
  \begin{enumerate}

  \item
  At the end of the first line: $il \leftarrow il + 1$

  \item
  At the end: $il \leftarrow il - 1$
  \end{enumerate}

\item
In a case (of a cases function expression):
  \begin{enumerate}

  \item
  After the arrow ("\verb|=>|") line: $il \leftarrow il + 1$.

  \item
  At the end: $il \leftarrow il - 1$.
  \end{enumerate}

\item
In a type theorem:
  \begin{enumerate}

  \item
  After "\verb|=|" line: $il \leftarrow il + 2$.

  \item
  At the end: $il \leftarrow il - 2$.
  \end{enumerate}

\item
In a cases function expression which does not begin at the "=" line of a value
definition:
  \begin{enumerate}

  \item
  After the the paremeters line: $il \leftarrow il + 1$.

  \item
  At the end of the cases function expression: $il \leftarrow il - 1$.

  \end{enumerate}

\end{enumerate}

\newpage
\subsection{High level structure}

\subsubsection{Parsec library}


The parser was implemented using the \textbf{parsec} library \cite{parsec}.
Parsec is an industrial strength, monadic parser combinator library for
Haskell.  It can parse context-sensitive, infinite look-ahead grammars. It
achieves this with a polymorphic parser type with the following parameter
types:

\begin{itemize}
\item
  \textit{stream type:} The input type to the parser.
\item
  \textit{user state type:} Type of custom state added by the parser developer.
\item
  \textit{underlying monad type:} A custom monad type in case it is needed.
\item
  \textit{return type:}
  This is the type of the value that is built by the parser.
\end{itemize}
The library has a lot of very nice parsers and parser combinators.
The package description in hackage is in the following url:
\url{https://hackage.haskell.org/package/parsec}

\paragraph{In this parser}
\begin{itemize}
\item
  \textit{stream type:}
  \\
  In this parser this is String.
\item
  \textit{state type:}
  \\
  In this parser this is ParserState. It is defined in the parser.
  A paragraph explaining what it is follows.
\item
  \textit{underlying monad:}
  \\
  In this parser this is not used interestingly
  (Identity is the underlying monad).
\item
  \textit{return type:}
  \\
  This is the type of the value that is built during parsing.
  Every AST type is the return type of the correspoding (sub)parser.
\end{itemize}

\paragraph{State type of the parser: ParserState}\pend
Here's the actual code for it:
\begin{verbatim}
type IndentationLevel = Int
type InEqualLine = Bool
type ParserState = (IndentationLevel, InEqualLine)
\end{verbatim}
We need this state to enforce the indentation rules (of
\ref{subsubsec:indsys}).

\subsubsection{File structure}
The parser code is split into the following files:
\begin{itemize}
\item
\verb|ASTTypes.hs|: Definitions of abstract syntax tree types
\item
\verb|ShowInstances.hs|: String respresentations for each AST type
\item
\verb|Parsers.hs|: Parsers for each AST type
\item
\verb|Testing.hs|: Runs the parsers on the examples and prints the result
\end{itemize}
All of the above are written using the full grammar. The types correspond to
non-terminal symbols. The parsers try to parse a string into the correspoding
AST type. If the string is valid every terminal symbol is discarded unless it's
part of a literal or an identifier.

\subsection{Parser Examples}

In this section we show how the types and the parsers are derived from the
grammar with some examples. We begin with a grammar rule and we create the
AST type and the parser that parses it.

\subsubsection{Parser Class and Example 0: Literal}

We have the Parser type which is polymorphic in the return
type with a stream type of String and a state type of ParserState:
\begin{verbatim}
type Parser = Parsec String ParserState

\end{verbatim}
We create the polymorphic value "parser" with the "HasParser" class
so that all the parsers have the name "parser" irrespective of the particular
type they are parsing:
\begin{verbatim}
class HasParser a where
  parser :: Parser a

\end{verbatim}
We begin with the very simple example of the literal with the following
grammar rule:
\begin{grammar}
<literal> ::= <int-lit> | <real-lit> | <char-lit> | <string-lit>\\
\end{grammar}
The AST type for the literal is:
\begin{verbatim}
data Literal =
  Int Int | R Double | Ch Char | S String

\end{verbatim}
And here is the parser for the literal which is defined as an instance of
the HasParser class. Inside we use the parsers for each particular literal
which are defined seperately:
\begin{verbatim}
instance HasParser Literal where
  parser =
    R <$> try parser <|> Int <$> parser <|> Ch <$> parser <|> S <$> parser <?>
    "Literal"

\end{verbatim}
The Parser type is a Functor so the "\verb|<$>|" (fmap) operator passes each
constructor inside the particular parser and the "\texttt{<|>}" operator means
"this parser or that parser". Finally, the "\verb|<?>|" operator means "if
all the parsers fail show this in the error message".

\newpage
\subsubsection{Example 1: List}

The grammar rule for the list is the following:
\begin{grammar}
<list> ::= `[' [ `\ ' ] [ <line-expr-or-unders> ] [ `\ ' ] `]' \\
\end{grammar}
The AST type for the list is:
\begin{verbatim}
newtype List = L (Maybe LineExprOrUnders)

\end{verbatim}
And the parser for the list is:
\begin{verbatim}
instance HasParser List where
  parser =
    L <$> (char '[' *> opt_space_around (optionMaybe parser) <* char ']')

\end{verbatim}
Here we use the "\verb|*>|" and "\verb|<*|" operators which parse both of
the parsers that they have as operands (from left to right) but only keep the
result of the parser that the "point" to. "opt_space_around" parses one space
optionally on each side of the text parsed by the argument parser and returns
what was parsed by the argument parser.  "optionMaybe" is defined in the
library and it optionally parses what its argument parser parses. If the
argument parser succeeds at parsing the it returns a Just $<$whatever was
parsed$>$, whereas if it fails it returns Nothing.

\subsubsection{Example 2: Change}

The grammar rule for the "change" expression is the following:
\begin{grammar}
<dot-change> ::=
`.change{' [ `\ ' ] <field-change> ( <comma> <field-change> )* [ `\ ' ] `}'
\\
\end{grammar}
The AST type for the "change" expression is:
\begin{verbatim}
newtype DotChange = DC (FieldChange, [FieldChange])

\end{verbatim}
And the parser for the "change" expression is:
\begin{verbatim}
instance HasParser DotChange where
  parser =
    DC <$>
      (try (string ".change{") *> opt_space_around field_changes_p <* char '}')
    where
    field_changes_p :: Parser (FieldChange, [FieldChange])
    field_changes_p = field_change_p +++ many (comma *> field_change_p)

\end{verbatim}
Here we use the "\verb|+++|" operator which is defined in the parser. It takes
two parsers as operands and creates a parser that uses them sequentially and
puts the two results in a tuple. "field_change_p" parses a single field change.
"many" is defined in the library, it parses with the argument parser as many
times as possible and puts the results in a list (the Kleene star of the parser
world).  Lastly, we also use the "try" parser combinator (defined in the
library) which is used to avoid comsuming any input if the parser fails. For
example if the input was "\verb|.change ...|", the parser would fail because we
have a space instead of an opening bracket.  Without the "try" combinator it
would consume the "\verb|.change|" part and leave only "\verb| ...|" after
failing. This would result in the PostFuncAppEnd alternative parser below
failing as well (the parser after "\verb|DC1 <$>|" is the Change parser).
Whereas with the "try" combinator we avoid this problem and the parser after
"\verb|PFsMDC <$>|" successfully parses ".change" as an alternative postfunc
(get the "change" field of a tuple type value).

\begin{verbatim}
instance HasParser PostFuncAppEnd where
  parser = DC1 <$> parser <|> PFsMDC <$> many1 parser +++ optionMaybe parser
\end{verbatim}


\subsubsection{Example 3: Value Definition}

The grammar rule for the value definition is the following:
\begin{grammar}
<value-def> ::= ""\\
<indent> <identifier>
( [ `\ ' ] `:' [ `\ ' ] | <nl> <indent> `:\ ' ) <type>  \\
<nl> <indent> `=\ ' <value-expr> [ <where-expr> ]
\\
\end{grammar}
The AST type for the value definition is:
\begin{verbatim}
newtype ValueDef = VD (Identifier, Type, ValueExpr, Maybe WhereExpr)

\end{verbatim}
And the parser for the value definition is:
\begin{verbatim}
instance HasParser ValueDef where
  parser =
    indent *> parser >>= \identifier ->

    increase_il_by 1 >>

    has_type_symbol *> parser >>= \type_ ->
    nl_indent *> string "= " *>

    increase_il_by 1 >> we_are_in_equal_line >>

    parser >>= \value_expr ->

    we_are_not_in_equal_line >>

    optionMaybe (try parser) >>= \maybe_where_expr ->

    decrease_il_by 2 >>

    return (VD (identifier, type_, value_expr, maybe_where_expr))

\end{verbatim}
In this example we see how the state of the parser is used to enforce the
indentation rules. The "indent" parser parses $<$2 * the indentation level$>$
spaces, getting the indentation level from the state (see Indentation System
\ref{subsubsec:indsys}).  "increase_il_by" and "decrease_il_by" have a Parser
type but they don't actually parse anything, they are "parsers" that only
update the indentation level (but can also be combined with other parsers).
They are used as described in rule 2 of the Indentation System
(\ref{subsubsec:indsys}). "has_type_symbol" parses the following part of
the grammar rule:
( [ \lit{\ } ] \lit{:} [ \lit{\ } ]
\verb||| \synt{nl} \synt{indent} \lit{:\ }).
"we_are_in_equal_line" and "we_are_not_in_equal_line" change the state
to enforce rule 6 of the Indentation System.

\newpage
\section{Translation to \H}

\subsection{High Level Overview}

To avoid rewriting the whole \H\ type system, lcases is translated directly
to \H\ without any semantic analysis, except for the bare minimum that is
required for the translation. The high level phases for the translation are
the following:

\begin{itemize}
\item
\textbf{Collect}\\\\
In this phase we traverse the AST and following are collected:

\begin{itemize}
\item
All the "\texttt{naked case}" identifiers of all the \texttt{or_types} in the
program, where a naked case is a case which does not contain an internal value
(e.g. \texttt{no_value} as opposed to \texttt{the_value:_}).
\item
All the field identifiers of the all the \texttt{tuple_types} in the program.
\end{itemize}

The above are used in the preprocess phase.

\item
\textbf{Preprocess}\\\\
In this phase the AST generated by the parser is traversed and tweaked
according to the following rules:

\begin{itemize}
\item
If an identifier of a naked case is found in a value expression, add a 'C' (for
constructor) in the front. This is going to be needed because \texttt{or_type}
cases are translated to data constructors in \H. Data constructors need to
start with an upper case letter whereas \texttt{or_type} cases don't
(this is not needed for the \texttt{or_type} cases with internal values
because we can identify them on the spot from the colon and handle them
appropriately in the translate phase).
\item
If a field identifier is found in a value expression which is a subexpression
of a "\verb|.change|" expression than the identifier is converted to a postfix
function with the same argument as the "\verb|.change|". For example, if the
AST has a part representing the expression below, it is going be
converted as follows:\\
\texttt{x.change\{f1 = f1 + 1\}} $\Longrightarrow$
\texttt{x.change\{f1 = x.f1 + 1\}}\\\\
The same applies to special identifiers:\\
\texttt{x.change\{1st = 1st + 1\}} $\Longrightarrow$
\texttt{x.change\{1st = x.1st + 1\}}
\end{itemize}

\item
\textbf{Translate}\\\\
In this phase the preprocessed AST is directly translated to \H. The details
of how this is done are described in the following section.
\end{itemize}

\newpage
\subsection{Translation Phase}

For the translation phase every type of the AST has its implementation of
one of the following 3 polymorphic functions:

\begin{itemize}
\item
\texttt{to_haskell}:\\\\
This function is for the AST types that need no state and can be directly
translated to \H.
\item
\texttt{to_hs_wpn}: (to haskell with parameter number) \\\\
This function is for the AST types that implicitly introduce extra
parameters when translated to \H\ and therefore need a state to keep track
of how many parameters have been introduced.
\item
\texttt{to_hs_wil}: (to haskell with indentation level)\\\\
This function is for the AST types that need indentation level information to
be correctly indented when translated to \H. Therefore, a state to keep track
of the indetation level is needed.\\
\end{itemize}
Here are the corresponding classes that define the above functions:

\begin{itemize}
\item
\begin{verbatim}
type Haskell = String

class ToHaskell a where
  to_haskell :: a -> Haskell
\end{verbatim}

\item
\begin{verbatim}
type WithParamNum = State Int

class ToHsWithParamNum a where
  to_hs_wpn :: a -> WithParamNum Haskell
\end{verbatim}

\item
\begin{verbatim}
type WithIndentLvl = State Int

class ToHsWithIndentLvl a where
  to_hs_wil :: a -> WithIndentLvl Haskell
\end{verbatim}
\end{itemize}
In the following sections we dive into more detail for particular types.

\newpage
\subsection{Translation Phase: Basic Expressions}

\subsubsection{Literals and Identifiers}
\begin{itemize}
\item
\textbf{Literals}\\\\
Literals are kept the same as they were parsed except for number
literals. Number literals have explicit type annotations (when they are not a
case of a "cases" function expression), so that they have type \texttt{Int}
or \texttt{Float} and not \texttt{Num a => a} or \texttt{Fractional p => p}
for the \H\ compiler.

\item
\textbf{Identifiers}

\begin{itemize}
\item
\textit{Examples}\\\\
\texttt{x1} \lra \texttt{x1} \\
\texttt{apply(_)to_all_in(_)} \lra \texttt{apply'to_all_in'} \\
\texttt{(_)to_string} \lra \texttt{a'to_string} \\
\texttt{f(_, _, _)} \lra \texttt{f'''}\\

\item
\textit{Description}\\\\
For identifiers all underscores in parenthesis are replaced by an equal amount
of single quotes. If there is a parenthesis in the beginning of the identifier,
the letter 'a' is prepended to make it a valid \H\ identifier.

\end{itemize}

\end{itemize}

\subsubsection{Parenthesis, Tuples and Lists}
\label{subsubsec:parentuplist}

\begin{itemize}
\item
\textbf{Parenthesis}\\\\
The internal expression is put in parenthesis.\\

\item
\textbf{Tuples}
\begin{itemize}
\item
\textit{Examples}\\\\
\texttt{(x, y)} \lra \texttt{ft2(x, y)} \\
\texttt{(_, 3.14, _)} \lra \verb|(\(pA0, pA1) -> ft3(pA0, 3.14, pA1))|\\
\texttt{(_, _, "Hello from 3rd field")} \lra
\verb|(\(pA0, pA1) -> ft3(pA0, pA1, "Hello from 3rd field"))|\\

\item
\textit{Description}
\paragraph{ft\textit{n} function}\pend
Every tuple is passed through the \textbf{ft\textit{n}} function where
\textbf{\textit{n}} is the size of the tuple. This is a polymorphic function
defined by the following class (for \textbf{\textit{n}} = 2 and with similar
classes for 3, 4 and 5):
\begin{verbatim}
class FromTuple2 a b c | c -> a b where
  ft2 :: (a, b) -> c
\end{verbatim}
This is because the same tuple may be of a \texttt{product type} or of a
\texttt{tuple_type} depending on where it appears on the program.
For product types we have the following instance:
\begin{verbatim}
instance FromTuple2 a b (a, b) where
  ft2 = id
\end{verbatim}
And there is also a new instance automatically generated for every
\texttt{tuple_type} definition.\\
For example, for the following definition:
\begin{verbatim}
tuple_type Name
value (first_name, last_name) : String^2
\end{verbatim}
Along with the definition itself which is described in section
\ref{subsubsec:trantypedef} there will be the following instance:
\begin{verbatim}
instance FromTuple2 String String Name where
  ft2 = \(x1, x2) -> Name' x1 x2
\end{verbatim}
With this mechanism the program will type check correctly on both cases.

\paragraph{Parameters for the underscores}\pend
If any of the fields of the tuples contain underscores, we generate a new
parameter name  in place every underscore ("\texttt{pA\textit{<n - 1>}}" for
the n-th underscore) and at the end we prepend the parameters to make it a
function expression. "pA" stands for parameter and the A is uppercase to avoid
collisions with regular identifiers.

\paragraph{Big Tuples}\pend
For big tuples, the original lines are kept and split in the same way.
The \texttt{ft\textit{n}} function and the implicitly introduced parameters
have their own seperate lines at the top as shown in the following example:
\begin{verbatim}
( "Hey, I'm the first field and I'm also a pretty big string."
, "Hey, I'm the second field and I'm a smaller string."
, _
)
\end{verbatim}
Becomes:
\begin{verbatim}
\pA0 ->
ft3
( "Hey, I'm the first field and I'm also a pretty big string."
, "Hey, I'm the second field and I'm a smaller string"
, pA0
)
\end{verbatim}
All of these lines are also indented to the same column according to the
indentation level.

\end{itemize}

\item
\textbf{Lists}

\begin{itemize}
\item
\textit{Examples}\\\\
\texttt{[1.61, 2.71, 3.14]} \lra \texttt{[1.61, 2.71, 3.14]} \\
\texttt{[_, x, _]} \lra \verb|(\(pA0, pA1) -> [pA0, x, pA1])| \\

\item
\textit{Description}\\\\
Lists work the same way as tuples except for the fact that they don't need
the \texttt{ft\textit{n}} function.

\end{itemize}

\end{itemize}

\newpage
\subsubsection{Parenthesis Function Application}

\begin{itemize}
\item
\textit{Examples}\\\\
\texttt{f(x, y, z)} \lra \texttt{f'''(x, y, z)} \\
\texttt{f(x, _, _)} \lra \verb|(\(pA0, pA1) -> f'''(x, pA0, pA1))| \\
\texttt{(x)to_str} \lra \texttt{a'to_str(x)} \\
\texttt{apply(f)to_all_in(_)} \lra \verb|(\pA0 -> apply'to_all_in'(f, pA0))| \\

\item
\textit{Description}\\\\
For the parenthesis function application, we seperate the underlying function
identifier by putting quotes in the place of the arguments (and prepending an
'a' if needed) and we also collect the arguments in a tuple. From there, we can
apply the function to the argument tuple. If any argument is an underscore, it
implicitly introduces a parameter and it is treated similarly to how it is
treated in tuples (section \ref{subsubsec:parentuplist}).

\end{itemize}

\subsubsection{Prefix and Postfix Functions}


\begin{itemize}
\item
\textbf{Prefix functions}
\begin{itemize}
\item
\textit{Examples}\\\\
\texttt{the_value:42} \lra \texttt{Cthe_value((42 ::\ Int))} \\
\texttt{error:"this is an error message"} \lra
\texttt{Cerror("this is an error message")} \\
\texttt{the_value:_} \lra \verb|(\pA0 -> Cthe_value(pA0))| \\
\texttt{the_value:result:true} \lra \texttt{Cthe_value(Cresult(True))} \\

\item
\textit{Description}\\\\
Prefix functions are data constructors in \H\ which are introduced by the
translation of the relevant \texttt{or_type} definition. They are prepended
with an upper case 'C' to be valid data constructors and their argument is
placed in parenthesis. The argument can be an underscore, in which case a new
parameter name is put in its place, the appropriate lambda abstraction is
prepended and the whole expression is put in parenthesis to limit the scope of
the abstraction.
\end{itemize}

\item
\textbf{Postfix functions}

\begin{itemize}
\item
\textit{Examples}\\\\
\texttt{date.year} \lra \texttt{year(date)} \\
\texttt{tuple.1st} \lra \texttt{p1st(tuple)} \\
\texttt{info.date.year} \lra \texttt{year(date(info))} \\
\texttt{tuple.1st.2nd} \lra \texttt{p2nd(p1st(tuple))} \\

\item
\textit{Description}\\\\
Postfix functions are projection functions that are generated automatically by
the translation of the relevant \texttt{tuple_type} defintion or they are the
projection functions for product types (\texttt{_.1st _.2nd etc}). They are
translated into regular \H\ functions (like the record accessor functions of
\H\ are) with their argument in parenthesis. For \texttt{_.1st _.2nd etc} a 'p'
for "projection" is prepended to make it a valid \H\ function.
\\\\
The projection functions for product types are polymorphic and work on tuples
of any size (in principle, for size $\leq$ 5 for now). This is achieved by
making them polymorphic through the following class (for p1st and similar
classes for the rest):
\begin{verbatim}
class IsFirst' a b | b -> a where
  p1st :: b -> a
\end{verbatim}
And the following instances:

\begin{verbatim}
instance IsFirst' a (a, b) where
  p1st = fst

instance IsFirst' a (a, b, c) where
  p1st = \(a, _, _) -> a

instance IsFirst' a (a, b, c, d) where
  p1st = \(a, _, _, _) -> a

...
\end{verbatim}
\end{itemize}

\item
\textbf{The ”.change” Function}

\begin{itemize}
\item
\textit{Examples}\\\\
\texttt{state.change\{counter = counter + 1\}}\\
\prarr
\texttt{state.change\{counter = state.counter + 1\}}\\
\lra
\texttt{c0counter(counter(state) !+ (1 ::\ Int)) state}
\\\\
\texttt{tuple.change\{1st = 1.61, 3rd = 3.14\}} \lra
\texttt{(c1st(1.61) .> c3rd(3.14)) tuple}
\\\\
\texttt{tuple.change\{x = _, y = _\}} \lra
\verb|(\(pA0, pA1) -> (c0x(pA0) .> c0y(pA1)) tuple)| \\

\item
\textit{Description}\\

For the "\texttt{.change}" function, similarly to the projection functions, a
change function (in \H) is introduced for every field of every
\texttt{tuple_type}. This function is the name of the field prepended by "c0"
where the 'c' stands for "change" and the '0' is there so that there can be no
collisions with other identifiers (numbers can only be at the end for lcases
identifiers). For the product type fields, the change functions are
"\texttt{c1st}", "\texttt{c2nd}", etc. The type of every change function
has the form:
\begin{verbatim}
FieldType -> TupleOrProdType -> TupleOrProdType
\end{verbatim}

For every assignment inside the braces we use the change function of the
assigned field with the expression of the assignment as the
\texttt{FieldType} argument. By doing this we have all the functions
of type:
\begin{verbatim}
TupleOrProdType -> TupleOrProdType
\end{verbatim}

we need to make the changes. We compose them all with the "\texttt{.>}"
operator, which is a predefined operator for function composition from left to
right (although the regular "\texttt{.}" should also do). Now we have one
function of type:
\begin{verbatim}
TupleOrProdType -> TupleOrProdType
\end{verbatim}

that does all the changes. We apply it on the argument of the
"\texttt{.change}" function and we are done.
\\\\
The change functions for product types work on tuples of any size
(in principle, for size $\leq$ 5 for now). This works with appropriate type
classes and instances, similarly to the way the \texttt{p1st}, \texttt{p2nd},
etc functions work. Unfortunately, because of this fact the regular braces
notation of \H\ for the same purpose cannot work, since \texttt{p1st},
\texttt{p2nd}, etc are polymorphic functions and not "record selectors".
\\\\
If an a variable is assigned to an underscore, new parameters are introduced
similarly to the way they are introduced to tuples.

\end{itemize}

\end{itemize}

\subsection{Translation Phase: Operators}

\subsubsection{Operators}

For each one of the lcases operators, a new \H\ operator is defined, for it to
be translated to. This allows for the use of the precedence and associativity
mechanism for new user defined operators in \H\ instead of having to implement
them from scratch in the parser, while also avoiding a lot of extra parentheses
that the latter approach would need. Function application and function
composition operators are defined as shown below. Every other operator is
defined by a type class that matches the type described in table
\ref{table:allops}. The implementaion of the operator for every combination of
types is defined by an appropriate instance. The examples for the addition
operator in the next page show the general structure.
\\

\begin{table}[ht]
\hspace{0.5cm}
\begin{minipage}{0.45\linewidth}\centering
\begin{tabular}{ |c|c| }
\hline
lcases operator & \H\ operator
\\
\hline
\hline
\verb|->| & \verb|&>|
\\
\hline
\verb|<-| & \verb|<&|
\\
\hline
\verb|o>| & \verb|.>|
\\
\hline
\verb|<o| & \verb|<.|
\\
\hline
\verb|^| & \verb|!^|
\\
\hline
\verb|*| & \verb|!*|
\\
\hline
\verb|/| & \verb|!/|
\\
\hline
\verb|+| & \verb|!+|
\\
\hline
\verb|-| & \verb|!-|
\\
\hline
\verb|==| & \verb|!==|
\\
\hline
\verb|!=| & \verb|!!=|
\\
\hline
\verb|>| & \verb|!>|
\\
\hline
\verb|<| & \verb|!<|
\\
\hline
\verb|>=| & \verb|!>=|
\\
\hline
\verb|<=| & \verb|!<=|
\\
\hline
\verb|&| & \verb|!&|
\\
\hline
\texttt{|} & \texttt{!|}
\\
\hline
\verb|;>| & \verb|!>>=|
\\
\hline
\verb|;| & \verb|!>>|
\\
\hline
\end{tabular}
\end{minipage}
\hspace{1cm}
\begin{minipage}{0.45\linewidth}

Precedence and associativity using \H:
\begin{verbatim}
infixl 9 &>
infixr 8 <&
infixl 7 .>, <.
infixr 6 !^
infixl 5 !*, !/
infixl 4 !+, !-
infix 3 !==, !!=, !>, !<, !>=, !<=
infixr 2 !&
infixr 1 !|
infixr 0 !>>=, !>>
\end{verbatim}

Function application/composition operators
\begin{verbatim}
(&>) :: a -> (a -> b) -> b
x &> f = f x

(<&) :: (a -> b) -> a -> b
f <& x = f x

(.>) :: (a -> b) -> (b -> c) -> a -> c
(.>) = flip (.)

(<.) :: (b -> c) -> (a -> b) -> a -> c
(<.) = (.)
\end{verbatim}
\end{minipage}
\end{table}
\newpage
\begin{Verbatim}[commandchars=\\\{\}]
# Type class for addition:
class A1And1Add_To1 a b c where
  (!+) :: a -> b -> c

# Some of the instances

# Adding 2 lists:
instance b \mytilde [a] => A1And1Add_To1 [a] [a] b where
  (!+) = (++)

# Adding any type 'a' with a list of 'a' s
instance b \mytilde [a] => A1And1Add_To1 a [a] b where
  (!+) = (:)

# Adding a String to a type 'a' with a Show instance without needing to call show
instance (Show a, b \mytilde String) => A1And1Add_To1 String a b where
  str !+ x = str ++ show x

\end{Verbatim}

\subsubsection{Operator Expressions}

\begin{itemize}
\item
\textit{Examples}\\\\
\verb|5 * 'a'| \lra \verb|(5 :: Int) !* 'a'|
\\\\
\verb|"Hello " + "World!"| \lra \verb|"Hello " !+ "World!"|
\\\\
\verb|_ - 1| \lra \verb|(\pA0 -> pA0 !- (1 :: Int))|
\\\\
\verb|_ + "string in the middle of the arguments" + _|
\\\lra
\verb|(\(pA0, pA1) -> pA0 !+ "string in the middle of the arguments" !+ pA1)|
\\

\item
\textit{Description}\\

In operator expressions, the operands are translated according to what operands
they are (described in their respective section) and the operators are
substituted by their respective \H\ operators. If an operand is an underscore
than a new lambda abstraction is introduced, similarly to how this is done for
tuples in \ref{subsubsec:parentuplist}.
\\\\
For big operator expression that span multiple lines, the lines are split the
same way they are split in the source file and they are indented all the same
according to the indentation level. If new lambda abstractrions are introduced,
they are all placed in a new line on the top, also indented the same. Again,
all of the above are done similarly to how they are done for tuples in
\ref{subsubsec:parentuplist}.

\end{itemize}

\newpage
\subsection{Translation Phase: Function Expressions}

\subsubsection{Regular Function Expressions}

\begin{itemize}
\item
\textit{Examples}\\\\
\verb|_| \lra \verb|_|
\\\\
\verb|_| \lra \verb|_|
\\\\
\verb|_| \lra \verb|_|
\\\\
\verb|_| \lra \verb|_|
\\\\

\item
\textit{Description}\\

\end{itemize}
\subsubsection{"cases" Function Expressions}

\subsection{Translation Phase: Value Definitions and "where" Expressions}

\subsubsection{Value Definitions}

\subsubsection{"where" Expressions}

\subsection{Translation Phase: Types}

\subsubsection{Type Expressions}

\subsubsection{Type Definitions}
\label{subsubsec:trantypedef}

\subsubsection{Type Nicknames}

\subsection{Translation Phase: Type Logic}

\subsubsection{Proposition Definitions}

\subsubsection{Theorems}

\subsection{Correspondaces with \H}

Here is the list of lcases constructs that have a \H\ correspondent:

\begin{itemize}
\item
Prefix Functions \r Data Constructors (with arguments)
\item
Postfix Functions (except "\verb|.change|" and "\verb|.<special-id>|")
\r
Accessor Functions generated by the Record Syntax
\item
"\verb|.change|" function
\r
Accessor Functions generated by the Record Syntax
\item
Operators:
\\
Operators are not actually compiled to the corresponding ones in \H\ because
the are some difference either in the precedence or in the types (in lcases
they are more general). Nevertheless, they are similar enough to be considered
"correspondances". Everything is compiled into function application in \H.
(TODO will it?)
\begin{itemize}
\item
Left function application operator "\verb|<-|" \r "\verb|$|" operator
\\
(althought it's not compiled to that due to difference in precedence)

\item
Left function application operator "\verb|->|" \r
"\verb|&|" operator from Data.Function
\\
(althought it's not compiled to that due to difference in precedence)

\item
Left function composition operator "\verb|<o|" \r "\verb|.|" operator

\item
Right function composition operator "\verb|o>|" \r
"\verb|>.>|" operator from GHC.Core.Map.Expr (or other equivalent ones from
other esoteric libraries, why \H?)

\item
"\verb|!=|" operator \r "\verb|\=|" operator

\item
"\verb|&|" operator \r "\verb|&&|" operator

\item
"\texttt{|}" operator \r "\texttt{||}" operator

\item
"\verb|;>|" operator \r "\verb|>>=|" operator

\item
"\verb|;|" operator \r "\verb|>>|" operator

\end{itemize}

\item
one parameter (one lambda abstraction) "\verb|a => ...|" \r
"\verb|\a -> ...|"

\item
multiple parameters (multiple lambda abstractions) "\verb|(a, b) => ...|" \r
"\verb|\a b -> ...|"

\item
"don't care" parameter
"\verb|* => ...|" \r "\verb|\_ => ...|"
\\
or one "don't care" in many
"\verb|(a, *, b) => ...|" \r "\verb|\a _ b => ...|"

\item
"cases" function expressions \r "\verb|\case|" syntax
(LambdaCase extension) \\
url: \url{https://typeclasses.com/ghc/lambda-case}

\item
"value definitions" are similar. In lcases the type and the expression are in
one definition. This is technically possible in \H\ too, but the type must be
after the expression and sometimes the expression must be put in parentheses
for the type to extend to the whole expression. "value definitions" are
compiled into a type annotation \textbf{and} a value definition in \H.

\item
"where" expressions are also similar. In lcases the values defined in a
"where" expression can "see" the parameters of the outer expression when
it is a function. "where" expressions in lcases are actually compiled to
"let-in" expressions in \H.

\item
Type variables are almost the same. In lcases, there syntactic separation of ad
hoc type variables (constrained type variables in \H) and parametric type
variables whereas in \H\ they are all lower case letters.

\item
The condition (constaint in \H) arrow: "\verb|-->|" \r "\verb|=>|"

\item
The product type: "\verb|A x B x C|" \r "\verb|(A, B, C)|"

\item
Type definitions \r "data" statements

\item
Type nicknames \r "type" statements

\item
Atomic type propositions \r Type classes

\item
Type theorems \r Type class instances

\end{itemize}

\section{Running Examples}

\section{Conclusion}

\newpage
\printbibliography

\end{document}
